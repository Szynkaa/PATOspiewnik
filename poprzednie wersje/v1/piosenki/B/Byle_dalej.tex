\begin{piosenka}{Byle dalej -- Andrzej Korycki}
	
\refrenspace Byle dalej i dalej, i dalej, i dalej, i dalej\ldots & d G C \\ 
\refrenspace Byle dalej i dalej, i dalej, i dalej, i dalej\ldots & d G C \\[\zwrotkaspace]

W nocy na przystanku wsiadam, & d G \\ 
W tłumie zapomnianych dłoni. & C a \\
Razem z deszczem na mnie spada, & d G \\
Sen wyśniony, wymarzony\ldots & C C$^7$ \\[\zwrotkaspace]

Znowu wieczór, powrót z pracy, & d G \\ 
W szybie cień zmęczonej twarzy & C a \\
I latarni światła spacer. & d G \\ 
Chcę inaczej żyć\ldots \ Inaczej. & F C \\[\zwrotkaspace] 

\refrenspace Móc nie słuchać cudzych rad, & C \\ 
\refrenspace Byle w uszach szumiał wiatr\ldots & d \\
\refrenspace Z wędrującą falą gnać, & G F \\
\refrenspace Byle dalej\ldots & C \\[\zwrotkaspace]

\refrenspace Nie żegnając dawnych dróg, & C \\ 
\refrenspace W żaglach sen odnaleźć mógł & d \\
\refrenspace Aż po brzegów zawołanie: & G F \\ 
\refrenspace Byle dalej\ldots & C \\[\zwrotkaspace]

\refrenspace Byle dalej i dalej, i dalej, i dalej, i dalej\ldots & d G C \\ 
\refrenspace Byle dalej i dalej, i dalej, i dalej, i dalej\ldots & d G C \\[\zwrotkaspace]

Zakołysał się autobus, & d G \\ 
Na pustkowiu drzwi otwarte. & C a \\
Może złapię w kroplach deszczu, & d G \\ 
Chwycę prawdę, że odnajdę\ldots & C C$^7$ \\[\zwrotkaspace]

W nocy sny wzburzone biją & d G \\ 
W odchodzących burty kutrów, & C a \\
Ku wzburzonym pędząc wichrom. & d G \\ 
Może minie czas mych smutków? & F C \\[\zwrotkaspace] 
	
\refrenspace Móc nie słuchać cudzych rad\ldots \\[\zwrotkaspace]

\refrenspace Byle dalej i dalej, i dalej, i dalej, i dalej\ldots & d G C \\ 
\refrenspace Byle dalej i dalej, i dalej, i dalej, i dalej\ldots & d G C \\[\zwrotkaspace]
	
	
\end{piosenka}