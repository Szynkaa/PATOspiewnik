{\small \begin{piosenka}{Baranek -- Kult}
Ach Ci ludzie, to brudne świnie & A \\
Co napletli o mojej dziewczynie & d \\
Jakieś bzdury o jej nałogach & A \\
To po prostu litość i trwoga & d \\
Tak to bywa gdy ktoś zazdrości & D \\
Kiedy brak mu własnej miłości & g \\
Plotki płodzi, mnie nie zaszkodzi żadne obce zło & A d \\
Na mój sposób widzieć ją & A d \\ [\zwrotkaspace]

\refrenspace Na głowie kwietny ma wianek & A d \\
\refrenspace W ręku zielony badylek & A d \\
\refrenspace A przed nią bieży baranek & g D \\
\refrenspace A nad nią lata motylek & A d \\ [\zwrotkaspace]

Krzywdę robią mojej panience & A \\
Opluć chcą ją podli zboczeńcy & d \\
Utopić chcą ją w morzu zawiści & A \\
Paranoicy, podli sadyści & d \\
Utaplani w brudnej rozpuście & D \\
A na gębach fałszywy uśmiech & g \\
Byle zagnać do swego bagna, ale wara wam & A d \\
Ja ją przecież lepiej znam & A d \\ [\zwrotkaspace]

\refrenspace Na głowie kwietny ma wianek\ldots \\ [\zwrotkaspace]

Znów widzieli ją z jakimś chłopem & A \\
Znów wyjechała do St. Tropez & d \\
Znów męczyła się, Boże drogi & A \\
Znów na jachtach myła podłogi & d \\
Tylko czemu ręce ma białe & D \\
Chciałem zapytać, zapomniałem & g \\
Ciało kłoniąc skinęła dłonią wsparła skroń o skroń & A d \\
Znów zapadłem w nią jak w toń & A d \\ [\zwrotkaspace]

\refrenspace Na głowie kwietny ma wianek\ldots \\ [\zwrotkaspace]

Ech, dziewczyna pięknie się stara & A \\
Kosi pieniądz, ma jaguara & d \\
Trudno pracę z miłością zgodzić & A \\
Rzadziej może do mnie przychodzić & d \\
Tylko pyta kryjąc rumieniec & D \\
Czemu patrzę jak potępieniec & g \\
Czemu zgrzytam, kiedy się pyta czy ma ładny biust & A d \\
Czemu toczę pianę z ust & A d \\

\end{piosenka}  }