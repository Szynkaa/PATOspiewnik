\begin{piosenka}{Ballada o świętym Mikołaju -- Andrzej Wierzbicki}
W rozstrzelanej chacie & a G E \\
Rozpaliłem ogień, & a G a \\
Z rozwalonych pieców & a G E \\
Pieśni wyniosłem węgle. & F E \\[\zwrotkaspace]

Naciągnąłem na drzazgi gontów & a C \\
Błękitną płachtę nieba & d E \\
Będę malować od nowa & a d C E a \\
Wioskę w dolinie. & d E a \\[\zwrotkaspace]

\akordy{G}\\[\zwrotkaspace]

\refrenspace Święty Mikołaju, opowiedz jak tu było, & C G C E$^7$ \\
\refrenspace Jakie pieśni śpiewano? & a d C E a \\
\refrenspace Gdzie się pasły konie? & d E a \\[\zwrotkaspace]

\refrenspace Święty Mikołaju, opowiedz jak tu było, & C G C E$^7$ \\
\refrenspace Jakie pieśni śpiewano? & a d C E a \\
\refrenspace Gdzie się pasły\ldots & d E a \\[\zwrotkaspace]

A on nie chce gadać & a G E \\
Ze mną po polsku & a G a \\
Z wypalonych źrenic & a G E \\
Tylko deszcze płyną. & F E \\[\zwrotkaspace]

Hej ślepcze, nauczę & a C \\
Swoje dziecko po łemkowsku & d E \\
Będziecie razem żebrać & a d C E a \\
W malowanych wioskach. & d E a \\
$\Vert\ \times$ 2 \\[\zwrotkaspace]

\akordy{G}\\[\zwrotkaspace]

\refrenspace Święty Mikołaju\ldots
\end{piosenka}