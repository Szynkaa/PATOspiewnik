\begin{piosenka}[6mm]{Rzuć mnie mała -- Andrzej Korycki}

Rzuć mnie mała, a będę miał pretekst, & A \\
By zaraz w morze iść. & H$^7$ \\ 
Rzuć mnie mała, by naszła znów mnie chęć & E E$^7$ \\
Na morskich przygód kiść & A H$^7$ E \\[\zwrotkaspace]
	
Rzuć mnie mała, niech inny w twej szafie & A \\
Swój rozpakuje wór & H$^7$ \\ 
Rzuć mnie mała, bo tak jak na rafie tu & E E$^7$ \\
Tkwię, więc tnij ten sznur. & A F$^7$ E$^7$ A \\[\zwrotkaspace]
 
\refrenspace Ja do ciebie,  co tu kryć,  nie pasuję, (nie pasuję!) & D A \\
\refrenspace Bo ty w pionie lubisz być, a ja się kiwać chcę. & H$^7$ E$^7$ \\
\refrenspace Rzuć mnie mała, niech inny w twej szafie & A \\
\refrenspace Swój rozpakuje wór & H$^7$ \\ 
\refrenspace Rzuć mnie mała, bo tak jak na rafie tu & E E$^7$ \\
\refrenspace Tkwię, więc tnij ten sznur & A F$^7$ E$^7$ A \\
\refrenspace $\| \times 2$ \\[\zwrotkaspace]

Rzuć mnie mała, bo ja nie potrafię żyć & A \\
Jak lądowy szczur & H$^7$ \\ 
Rzuć mnie mała i z innym wieś zasiedl & E E$^7$ \\
Gdzieś u podnóży gór & A H$^7$ E \\[\zwrotkaspace]

Rzuć mnie mała, a cieszył się będę & A \\
Od rufy aż po dziób & H$^7$ \\ 
Rzuć mnie mała, i zrób to czym prędzej, & E E$^7$ \\
Bo jeśli nie - to ślub! & A F$^7$ E$^7$ A \\[\zwrotkaspace]

\refrenspace Choć do ciebie,  co tu kryć,  nie pasuję, (nie pasuję!) & D A \\
\refrenspace Bo ty w pionie lubisz być, a ja się kiwać chcę. & H$^7$ E$^7$ \\
\refrenspace Zgódź się mała, a cieszył się będę & A \\
\refrenspace Od rufy aż po dziób & H$^7$ \\ 
\refrenspace Zgódź się mała, i zrób to czym prędzej, & E E$^7$ \\
\refrenspace A wtedy jutro ślub! & A F$^7$ E$^7$ A \\
\refrenspace $\| \times 2$ \\[\zwrotkaspace]

\end{piosenka}