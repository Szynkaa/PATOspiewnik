{\small \begin{piosenka}[1mm]{Aleksander Siergiejewicz Puszkin -- Bułat Okudżawa}
Co było -- nie wróci, i szaty rozdzierać by próżno. & a E$^7$ a \\
Cóż, każda epoka ma własny obyczaj i ład\ldots & C G$^7$ C A$^7$ \\
A przecież mi żal, że tu, w drzwiach, nie pojawi się Puszkin -- & d G$^7$ a \\
tak chętnie bym dziś choć na kwadrans na koniak z nim wpadł. & a F$^7$ E$^7$ a \\[\zwrotkaspace]

Dziś już nie musimy piechotą się wlec na spotkanie -- & a E$^7$ a \\
i tyle jest aut, i rakiety unoszą nas w dal\ldots & C G$^7$ C A$^7$ \\
A przecież mi żal, że po Moskwie nie suną już sanie, & d G$^7$ a \\
i nie ma już sań, i nie będzie już nigdy, a żal! & a F$^7$ E$^7$ a \\[\zwrotkaspace]

Przyjmuję pojętny mój wiek, mego stwórcę i mistrza, & a E$^7$ a \\
ten trzeźwy mój wiek, doświadczony mój wiek pragnę czcić\ldots & C G$^7$ C A$^7$ \\
A przecież mi żal, że jak dawniej śnią nam się bożyszcza & d G$^7$ a \\
i jakoś tak jest, że gotowiśmy czołem im bić. & a F$^7$ E$^7$ a \\[\zwrotkaspace]

No cóż, nie na darmo zwycięstwem nasz szlak się uświetnił, & a E$^7$ a \\
i wszystko już jest -- cicha przystań, nonajron i wikt\ldots & C G$^7$ C A$^7$ \\
A przecież mi żal, że nad naszym zwycięstwem niejednym & d G$^7$ a \\
górują cokoły, na których nie stoi już nikt. & a F$^7$ E$^7$ a \\[\zwrotkaspace]

Co było -- nie wróci; wychodzę wieczorem na spacer & a E$^7$ a \\
i nagle spojrzałem na Arbat i -- ach, co za gość! -- & C G$^7$ C A$^7$ \\
rżą konie u sań, Aleksander Siergiejewicz przechadza się, & d G$^7$ a \\
ach, głowę bym dał, że już jutro wydarzy się coś! & a F$^7$ E$^7$ a \\[\zwrotkaspace]

\end{piosenka} }