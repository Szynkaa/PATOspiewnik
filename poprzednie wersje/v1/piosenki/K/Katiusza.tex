{\small \begin{piosenka}{Катюша}
	
Расцветали яблони и груши, & a E$^7$ \\
Поплыли туманы над рекой, & E$^7$ a \\
Выходила на берег Катюша & A$^7$ d a \\
На высокий берег на крутой! $\| \times 2$ & d a E$^7$ a \\[\zwrotkaspace]

Выходила, песню заводила & a E$^7$ \\
Про степного сизого орла, & E$^7$ a \\
Про того, которого любила, & A$^7$ d a \\
Про того, чьи письма берегла. $\| \times 2$ & d a E$^7$ a \\[\zwrotkaspace]

Ой, ты песня, песенка девичья, & a E$^7$ \\
Ты лети за ясным солнцем вслед & E$^7$ a \\
И бойцу на дальнем пограничье & A$^7$ d a \\
От Катюши передай привет. $\| \times 2$ & d a E$^7$ a \\[\zwrotkaspace]

Пусть он вспомнит девушку простую, & a E$^7$ \\
Пусть услышит, как она поёт, & E$^7$ a \\
Пусть он землю бережёт родную, & A$^7$ d a \\
А любовь Катюша сбережёт. $\| \times 2$ & d a E$^7$ a \\[\zwrotkaspace]

Расцветали яблони и груши, & a E$^7$ \\
Поплыли туманы над рекой, & E$^7$ a \\
Выходила на берег Катюша & A$^7$ d a \\
На высокий берег на крутой!	$\| \times 2$ & d a E$^7$ a \\[\zwrotkaspace]
	
\end{piosenka}}\\
{\small 
	\begin{tabular}{l}
		\textit{Rascwietali jabłani i gruszy,} \\
		\textit{Papłyli tumany nad riekoj; }\\
		\textit{Wychadiła na bierieg Katiusza,} \\
		\textit{Na wysokij, bierieg na krutoj.} \\[1mm]
		
		\textit{Wychadiła, piesniu zawadiła} \\
		\textit{Pra stiepnowa sizawa arła,} \\
		\textit{Pra tawo, katorawa liubiła,} \\
		\textit{Pra tawo, ćji pis'ma bieriegła.} \\[1mm]
		
		\textit{Oj, ty piesnia, piesienka diewićja,} \\
		\textit{Ty leti za jasnym soncem wslied,} \\
		\textit{I bajcu na dalniem pogranićje} \\
		\textit{Ot Katiuszy pieriedaj priwiet.} \\[1mm]
		
		\textit{Pust' on wspomnit diewuszku prastuju,} \\
		\textit{Pust' usłyszyt, kak ana pajot,} \\
		\textit{Pust' on ziemlu bierieżot radnuju} \\
		\textit{A liubow' Katiusza sbierieżot.} \\[1mm]
		
		\textit{Rascwietali jabłani i gruszy\ldots} \\
		
	\end{tabular}

 }