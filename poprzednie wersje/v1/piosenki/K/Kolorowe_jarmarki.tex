\begin{piosenka}{Kolorowe jarmarki -- Janusz Laskowski}
Kiedy patrzę hen za siebie & a \\
W tamte lata co minęły & d \\
Czasem myślę co przegrałam & G \\
Ile diabli wzięli & C E$^7$ \\
Co straciłam z własnej woli & d G \\
Ile przeciw sobie & C a \\
Co wyliczę to wyliczę & E \\
Ale zawsze wtedy powiem, że najbardziej mi żal: & E$^7$ a \\[\zwrotkaspace]

\refrenspace Kolorowych jarmarków, blaszanych zegarków & A$^7$ d G C \\
\refrenspace Pierzastych kogucików, baloników na druciku & a d E$^7$ a A$^7$ \\
\refrenspace Motyli drewnianych, koników bujanych & d G C \\
\refrenspace Cukrowej waty i z piernika chaty & a d E$^7$ a A$^7$ \\[\zwrotkaspace]

Gdy w dzieciństwa wracam strony & a \\
Dobre chwile przypominam & d \\
Mego miasta słyszę strony & G \\
Czy ktoś czas zatrzymał & C E$^7$ \\
I gdy pytam cicho siebie & d G \\
Czego żal dziś tobie & C a \\
Co wyliczę to wyliczę & E \\
Ale zawsze wtedy powiem, ze najbardziej mi żal: & E$^7$ a \\[\zwrotkaspace] 

\refrenspace Kolorowych jarmarków\ldots \\

\end{piosenka}