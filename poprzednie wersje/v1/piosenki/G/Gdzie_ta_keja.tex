\begin{piosenka}{Gdzie ta keja -- Jerzy Porębski}
Gdyby tak ktoś przyszedł i powiedział: & a \\
Stary, czy masz czas? & G a \\
Potrzebuje do załogi jakąś nową twarz & C G G$^7$ C \\
Amazonka, Wielka Rafa, oceany trzy & C C$^7$ F d \\
Rejs na całość, rok, dwa lata, to powiedziałbym: & a E E$^7$ a \\[\zwrotkaspace]

\refrenspace Gdzie ta keja, a przy niej ten jacht? & a E$^7$ a \\
\refrenspace Gdzie ta koja wymarzona w snach? & C G C \\
\refrenspace Gdzie te wszystkie sznurki od tych szmat? & g A$^7$ d A$^7$ d \\
\refrenspace Gdzie ta brama na szeroki świat? & a E$^7$ a \\[\zwrotkaspace]

\refrenspace Gdzie ta keja, a przy niej ten jacht? & a E$^7$ a \\
\refrenspace Gdzie ta koja wymarzona w snach? & C G C \\
\refrenspace W każdej chwili płynę w taki rejs & g A$^7$ d A$^7$ d \\
\refrenspace Tylko gdzie to jest, no gdzie to jest? & a E$^7$ a \\[\zwrotkaspace]

Gdzieś na dnie starej szafy leży ostry nóż & a G a \\
Stare dżinsy wystrzępione impregnuje kurz & C G G$^7$ C  \\
W kompasie igła zardzewiała, lecz kierunek znam & C C$^7$ F d \\
Biorę wór na plecy i przed siebie gnam & a E E$^7$ a \\[\zwrotkaspace]

\refrenspace Gdzie ta keja\ldots \\[\zwrotkaspace]

Przeszły lata zapyziałe, rzęsą zarósł staw & a G a \\
Na przystani czółno stało -- kolorowy paw & C G G$^7$ C  \\
Zaokrągliły się marzenia, wyjałowiał step & C C$^7$ F d \\
Dalej marzy o załodze ten samotny łeb & a E E$^7$ a \\[\zwrotkaspace]

\refrenspace Gdzie ta keja\ldots \\
\end{piosenka}
