\begin{piosenka}{Gajowy na haju -- Artur Andrus}
Raz staruszek spacerując wokół & e A$^7$ e A$^7$ \\
zagajnika ,,Niedźwiedzia Pieczara'' & e A$^7$ H$^7$ \\ 
ujrzał nagle cztery pory roku. & e A$^7$ e \\ 
Cztery pory -- i to wszystkie naraz. & C H$^7$ e A$^7$ \\ [\zwrotkaspace]

\refrenspace I podreptał do domu po dróżce & C D G e \\
\refrenspace przydeptując łeb leżącej kobrze & C D G e \\
\refrenspace i powiedział swej żonie staruszce & C D G e \\
\refrenspace -- Oj, Halino, coś ze mną niedobrze!  & C H$^7$ e A$^7$ e H$^7$ \\ [\zwrotkaspace] 

A Halina zmartwiła się trochę & e A$^7$ e A$^7$ \\
i zrobiła napar z kwiatu mięty & e A$^7$ H$^7$ \\ 
-- Oj, staruszku, toż to przez te prochy. & e A$^7$ e \\ 
Weźże wyrzuć te cholerne skręty! & C H$^7$ e A$^7$ \\ [\zwrotkaspace]

\refrenspace A ty jeszcze popijasz to piwem & C D G e \\
\refrenspace i wychodzisz do roboty w gaju. & C D G e \\
\refrenspace Już podobno nadali ci ksywę & C D G e \\
\refrenspace ,,Postrach Lasu -- Gajowy Na Haju''.  & C H$^7$ e A$^7$ e H$^7$ \\ [\zwrotkaspace] 

A to wszystko zdarzyło się w maju & e A$^7$ e A$^7$ \\
gdy pogoda była całkiem dobra\ldots & e A$^7$ H$^7$ \\ 
Państwo pewnie się zastanawiają & e A$^7$ e \\ 
jak w tym lesie znalazła się kobra. & C H$^7$ e A$^7$ \\ [\zwrotkaspace]

\refrenspace Nim w oklaskach zewrą Państwo dłonie & C D G e \\
\refrenspace jakoś się to wyjaśnić postaram. & C D G e \\
\refrenspace Tak naprawdę, to to był zaskroniec. & C D G e \\
\refrenspace Z tym, że duży. I że w okularach & C H$^7$ e A$^7$ e H$^7$ \\ [\zwrotkaspace] 
\end{piosenka}