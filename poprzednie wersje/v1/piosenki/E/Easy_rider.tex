\begin{piosenka_dluga}[4mm]{Easy Rider -- Krzysztof Daukszewicz}
	
I kiedy nic już nie miałem w mieście do roboty & e \\
Bo na większość poetów skończył się tu popyt & e \\
Wsiadłem w auto i rzekłem pora mi uciekać & a \\
Do tej Polski gdzie jeszcze kocha się człowieka & H$^7$ \\
Tam gdzie rowy przydrożne ubarwione mleczem & a \\
Zapraszają wędrowca ,,wstąpcie do miasteczek'' & H$^7$ \\[\zwrotkaspace]

\refrenspace Easy rider -- przeszło mi przez głowę & e a \\
\refrenspace Easy rider -- głupiec jednym słowem & e H$^7$ \\
\refrenspace Lecz ciągnęły mnie panny ciepłe jak poranek & a \\
\refrenspace Kiedy mleko skwaszone noszą mi na ganek & H$^7$ \\
\refrenspace Easy rider, mmm\ldots easy rider\ldots & e a e \\[\zwrotkaspace]

W miasteczku pierwszym zamknięty był jedyny hotel & e \\
Bo personel miał wolne właśnie w tę sobotę & e \\
A w prywatnym mieszkaniu drzwi otworzył blondyn & a \\
I zapytał mnie z miejsca „jakie masz poglądy?” & H$^7$ \\
Sprawiedliwość i prawda to jest dla mnie wszystko & a \\
Wtedy padła odpowiedź ,,zjeżdżaj aktywisto!'' & H$^7$ \\[\zwrotkaspace]

\refrenspace Easy rider -- przeszło mi przez głowę & e a \\
\refrenspace Easy rider -- głupiec jednym słowem & e H$^7$ \\
\refrenspace Lecz ciągnęły mnie dalej wierzby malowane & a \\
\refrenspace I te nasze dziewczyny ładne jak z pisanek  & H$^7$ \\
\refrenspace Easy rider, mmm\ldots easy rider\ldots & e a e \\[\zwrotkaspace]

W następnym domku z ogródkiem miejski prokurator & e \\
Różom kolce przycinał równo ciął sekator & e \\
Przywitałem się grzecznie prosząc o mieszkanie & a \\
On zapytał mnie tylko ,,jakie ma Pan zdanie?'' & H$^7$ \\
Sprawiedliwość i prawda to jest dla mnie wszystko & a \\
Usłyszałem odpowiedź ,,odejdź ekstremisto!'' & H$^7$ \\[\zwrotkaspace]

\refrenspace Easy rider -- przeszło mi przez głowę & e a \\
\refrenspace Easy rider -- głupiec jednym słowem & e H$^7$ \\
\refrenspace Lecz ciągnęło mnie jeszcze do gościnnych wiosek & a \\
\refrenspace Gdzie częstują każdego miodem i bigosem  & H$^7$ \\
\refrenspace Easy rider, mmm\ldots easy rider\ldots & e a e \\[\zwrotkaspace]

Solidny dom z pruskiej cegły siatką ogrodzony & e \\
I na bramie tabliczka obcym wstęp wzbroniony & e \\
I na ganku gospodarz czerstwy jak bochenek & a \\
Wziął przywitał pytaniem co najbardziej cenie & H$^7$ \\
Sprawiedliwość i prawda to jest dla mnie wszystko & a \\
,,Burek bierz miastowego!'' będzie widowisko\ldots & H$^7$ \\[\zwrotkaspace]

\refrenspace Easy rider -- przeszło mi przez głowę & e a \\
\refrenspace Easy rider -- głupiec jednym słowem & e H$^7$ \\
\refrenspace Lecz ciągnęło mnie jeszcze w strony te dalekie & a \\
\refrenspace Gdzie tak swojsko nam pachnie sianem i człowiekiem  & H$^7$ \\
\refrenspace Easy rider, mmm\ldots easy rider\ldots & e a e \\[\zwrotkaspace]

A kiedy minął juz miesiąc w mej samotnej drodze & e \\
Gdzieś na szlaku zatrzymał pojazd mój wędrowiec & e \\
,,Sprawiedliwość i prawda'' rzekłem do rodaka & a \\
I był pierwszym co spytał „dobrze, ale jaka?” & H$^7$ \\
I podzielił się ze mną chlebem i kłopotem & a \\
To był też easy rider tylko na piechotę & H$^7$ \\[\zwrotkaspace]

\refrenspace Easy rider, mmm\ldots easy rider\ldots & e a e \\[\zwrotkaspace]	
	
\end{piosenka_dluga}	