{\small \begin{piosenka}{Major ponury}

Mgła wschodzi z lasu Panie Majorze & e D \\
Wiatr się po lesie snuje jak ptak & C G D \\
Już się szkopy nie tułają po borze & e h \\
Niejednego przez nas trafił szlak & C D \\
Jutro do wsi pewnie zajdziemy & e D \\
Pies nie szczeknie -- przecież my swoi & C G D \\
U mej matuli cokolwiek zjemy  & e h \\
Potem śpiewaniem do snu ukoi\ldots & C D \\[\zwrotkaspace]

\refrenspace I dobrze odpocznie nim odejdzie w góry & C D \\
\refrenspace Lecz co Pan Major taki ponury? & e D \\
\refrenspace $\| \times 2$ \\[\zwrotkaspace]

Do diabła ze śmiercią Panie Majorze & e D \\
Pan szedł z nią razem w 39-tym & C G D \\
Potem trza było sie z wojskiem łączyć & e h \\
I miecze ostrzyć daleko za morzem & C D \\
Myśmy czekali, bo wodza brakło & e D \\
I chytry zwierz, co walczy bez oka & C G D \\
Wieści przesłali słowo sie rzekło & e h \\
I biały orzeł z gór spikował\ldots & C D \\[\zwrotkaspace]

\refrenspace I w piersi wroga wbił swe pazury  & C D \\
\refrenspace Lecz co Pan Major taki ponury? & e D \\
\refrenspace $\| \times 2$ \\[\zwrotkaspace]

To nie był taki zwyczajny bój & e D \\
Lufa się zgrzała jak klucze od piekła & C G D \\
Mocno się wrzynał w kieszeni nabój & e h \\
I każda chwila jak wieczność się wlekła & C D \\
Strasznie Pan dostał Panie Majorze & e D \\
Jak mi Bóg miły nie mogło być gorzej & C G D \\
Krew się przelała przez głębokie rany & e h \\
Archanioł Michał otworzył bramy & C D \\[\zwrotkaspace]

\refrenspace Pozdrówcie ode mnie Świętokrzyskie Góry  & C D \\
\refrenspace Szepnął i skonał Major Ponury & e D \\
\refrenspace $\| \times 2$ \\[\zwrotkaspace]

\refrenspace Skonał i odszedł odnaleźć swe góry  & C D \\
\refrenspace Serca bohater Major Ponury\ldots & e D \\
\refrenspace $\| \times 2$ \\[\zwrotkaspace]

\end{piosenka} }