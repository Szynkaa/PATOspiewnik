\begin{piosenka}{Majster Bieda -- WGB}
\akordy{D G fis e A D D$^4$} \\[\zwrotkaspace]
Skąd przychodził, kto go znał & D G \\
Kto mu rękę podał kiedy & D G A \\
Nad rowem siadał, wyjmował chleb & D A \\
Serem przekładał i dzielił się z psem & fis h \\
Tyle wszystkiego, co z sobą miał & A G fis e \\
Majster Bieda & A D G fis e A D D$^4$\\[\zwrotkaspace]

Czapkę z głowy ściągał, gdy & D G \\
Wiatr gałęzie chylił drzewom & D G A \\
Śmiał się do słońca i śpiewał do gwiazd & D A \\
Drogę bez końca, co przed nim szła & fis h \\
Znał jak pięć palców, jak szeląg zły & A G fis e \\
Majster Bieda & A D G fis e A D D$^4$ \\[\zwrotkaspace]

Nikt nie pytał, skąd się wziął & D G \\
Gdy do ognia się przysiadał & D G A \\
Wtulał się w krąg ciepła jak w kożuch & D A \\
Znużony drogą wędrowiec Boży & fis h \\
Zasypiał długo gapiąc się w noc & A G fis e \\
Majster Bieda & A D G fis e A D D$^4$ \\[\zwrotkaspace]

Aż nastąpił taki rok & D G \\
Smutny rok, tak widać trzeba & D D$^7$ G A \\
Nie przyszedł Bieda zieloną wiosną & D A \\
Miejsce, gdzie siadał, zielskiem zarosło & fis h \\
I choć niejeden wytężał wzrok & A G \\
Choć lato pustym gościńcem przeszło & A G \\
Z rudymi liśćmi jesieni schedą & A G \\
Wiatrem niesiony popłynął w przeszłość & A G \\
Wiatrem niesiony popłynął w przeszłość & A G \\
Wiatrem niesiony popłynął w przeszłość & A G A \\
Majster Bieda & D G fis e A D \\
\end{piosenka}\\
\chord{t}{x,n,n,p2,p3,p3}{D$^4$} \chord{t}{x,n,n,p2,p1,p2}{D$^7$}