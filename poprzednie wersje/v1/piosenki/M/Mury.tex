{\footnotesize \begin{piosenka}[1mm]{Mury -- Jacek Kaczmarski}

On natchniony i młody był; ich nie policzyłby nikt, & e H$^7$ e H$^7$ \\
On dodawał im pieśnią sił, śpiewał, że blisko już świt. & C H$^7$ e C H$^7$ e \\
Świec tysiące palili mu, znad głów unosił się dym, & e H$^7$ e H$^7$ \\
Śpiewał, że czas, by runął mur, oni śpiewali wraz z nim: & C H$^7$ e C H$^7$ e H$^7$ \\[\zwrotkaspace]

\refrenspace Wyrwij murom zęby krat! & e H$^7$ e \\
\refrenspace Zerwij kajdany, połam bat, & e H$^7$ e \\
\refrenspace A mury runą, runą, runą & a e \\
\refrenspace I pogrzebią stary świat! & H$^7$ e \\[\zwrotkaspace]

Wkrótce na pamięć znali pieśń i sama melodia bez słów & e H$^7$ e H$^7$ \\
Niosła ze sobą starą treść, dreszcze na wskroś serc i głów. & C H$^7$ e C H$^7$ e \\
Śpiewali więc, klaskali w rytm, jak wystrzał poklask ich brzmiał & e H$^7$ e H$^7$ \\
I ciążył łańcuch, zwlekał świt -- a śpiewak wciąż śpiewał i grał & C H$^7$ e C H$^7$ e H$^7$ \\[\zwrotkaspace]

\refrenspace Wyrwij murom zęby krat\ldots \\[\zwrotkaspace]

Aż zobaczyli ilu ich, poczuli siłę i czas! & e H$^7$ e H$^7$ \\
I z pieśnią, że już blisko świt szli ulicami miast. & C H$^7$ e C H$^7$ e \\
Zwalali pomniki, rwali bruk, ten z nami, ten przeciw nam! & e H$^7$ e H$^7$ \\
Kto sam, ten nasz największy wróg -- a śpiewak także był sam & C H$^7$ e C H$^7$ e H$^7$ \\[\zwrotkaspace]

\refrenspace Patrzył na równy tłumów marsz, & e H$^7$ e \\
\refrenspace Milczał wsłuchany w kroków huk. & e H$^7$ e \\
\refrenspace A mury rosły, rosły, rosły, &a e \\
\refrenspace Łańcuch kołysał się u nóg\ldots & H$^7$ e \\[\zwrotkaspace]

\refrenspace Patrzy na równy tłumów marsz, & e H$^7$ e \\
\refrenspace Milczy wsłuchany w kroków huk. & e H$^7$ e \\
\refrenspace A mury rosną, rosną, rosną, &a e \\
\refrenspace Łańcuch kołysze się u nóg\ldots & H$^7$ e \\[\zwrotkaspace]

\end{piosenka} }