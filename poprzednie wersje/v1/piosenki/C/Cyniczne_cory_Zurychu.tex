\begin{piosenka_dluga}[6mm]{Cyniczne córy Zurychu -- Artur Andrus}

Ma stary tata Turek & G \\*
Sześć zaradnych córek: & G \\*
Ajsze, Baszak, Fatma, Dżanan, Burczu i Raszida. & G Gis Gis G \\*
Gdy się na córki złości, & G \\*
To w tej kolejności: & G \\*
Ajsze, Baszak, Fatma, Dżanan, Burczu i Raszida. & G Gis Gis G \\*
Którejś nocy wyszły z domu & G \\*
I uciekły po kryjomu & G \\*
Ajsze, Baszak, Fatma, Dżanan, Burczu i Raszida. & G Gis Gis G \\*
Dostał tata wieści, z których & G \\*
Wyszło, że są w mieście Zurych, & G \\*
Trafił tatę szlag & f G \\*
I zakrzyknął tak: & f G \\[\zwrotkaspace]
 
\refrenspace Cyniczne córy Zurychu & g \\*
\refrenspace Potępiam was wszystkie w czambuł, & F \\*
\refrenspace Cyniczne córy Zurychu, & Dis \\*
\refrenspace Płacze za wami Stambuł. & H$^7$ \\*
\refrenspace \akordy{G G} \\[\zwrotkaspace]

 
A już po latach paru & G \\*
Wyszły za Szwajcarów & G \\*
Ajsze, Baszak, Fatma, Dżanan, Burczu i Raszida. & G Gis Gis G \\*
I stało im się bliskie & G \\*
Jezioro Zuryskie & G \\*
Ajsze, Baszak, Fatmie, Dżanan, Burczu i Raszidzie. & G Gis Gis G \\*
Ich mężowie, śliczni chłopcy, & G \\*
Pięciu braci, jeden obcy: & G \\*
Simon, Lukas, Christian, Tobias, Jonas i Andreas. & G Gis Gis G \\*
Dostał tata Turek zdjęcia, & G \\*
Każde z podobizną zięcia. & G \\*
Porwał tatę szał & f G \\*
Włosy z głowy rwał & f G \\[\zwrotkaspace]

\refrenspace Cyniczne córy Zurychu\ldots \\[\zwrotkaspace]
 
Pracują wszystkie razem, & G \\*
Z Rosją handlują gazem & G \\*
Ajsze, Baszak, Fatma, Dżanan, Burczu i Raszida. & G Gis Gis G \\*
Statkami ślą przez Bosfor & G \\*
Uran, miedź i fosfor & G \\*
Ajsze, Baszak, Fatma, Dżanan, Burczu i Raszida. & G Gis Gis G \\*
W tajemnicy przed mężami & G \\*
Handlują też wyrzutniami & G \\*
Ajsze, Baszak, Fatma, Dżanan, Burczu i Raszida. & G Gis Gis G \\*
Tata zaś u życia kresu & G \\*
Został gwiazdą show-bussinesu, & G \\*
Radio ,,Stambuł 2'' & f G \\*
Na okrągło gra: & f G \\[\zwrotkaspace]
 
\refrenspace Cyniczne córy Zurychu\ldots \\[\zwrotkaspace]

\end{piosenka_dluga} 