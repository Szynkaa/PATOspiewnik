\begin{piosenka}[1.5mm]{Czy te oczy mogą kłamać -- Raz dwa trzy}
A gdy się zejdą, raz i drugi, & d a \\
Kobieta po przejściach, mężczyzna z przeszłością, & E a \\
Bardzo się męczą, męczą przez czas długi, & d a \\
Co zrobić, co zrobić z tą miłością? & E a \\
On już je zna, już zna te dziewczyny, & d a \\
Z poszarpanymi nerwami, co wracają nad ranem nie same, & d a d a \\
On już słyszał o życiu złamanym. & E a \\ 
Ona już wie, już zna tę historię, & d a \\
Że żona go nie rozumie, że wcale ze sobą nie śpią, & d a \\
Ona na pamięć to umie. & E a \\ 
A gdy przyjdzie zapomnieć i w pamięci to zatrzeć? & d a \\
Lepiej milczeć przytomnie i patrzeć. & E a \\[\zwrotkaspace]
 
\refrenspace Czy te oczy mogą kłamać? Chyba nie! & d \\
\refrenspace Czy ja mógłbym serce złamać? I te pe\ldots & a \\
\refrenspace Kiedyś to zrozumiesz sama,  To był błąd! & E \\
\refrenspace Czy te oczy mogą kłamać? Ależ skąd! & a E a \\[\zwrotkaspace]

\refrenspace Czy te oczy mogą kłamać? Chyba nie! & d \\
\refrenspace Czy ja mógłbym serce złamać? I te pe\dots & a \\
\refrenspace Gdy się farsa zmienia w dramat, nie gnam w kąt. & E \\
\refrenspace Czy te oczy mogą kłamać? Ależ skąd! & a E a \\[\zwrotkaspace]

A gdy się czasem w życiu uda & d a \\
Kobiecie z przeszłością, mężczyźnie po przejściach, & E a \\
Kąt wynajdują gdzieś u ludzi & d a \\
I łapią, i łapią trochę szczęścia. & E a \\
On zapomina na rok te dziewczyny & d a \\
Z bardzo długimi nogami, co wracają nad ranem nie same. & d a d a \\
Woli ciszę z radzieckim szampanem. & E a \\
Ona już ma, już ma taką pewność, & d a \\
O którą wszystkim wam chodzi, zasypia bez żadnych proszków, & d a d a \\
Wino w lodówce się chłodzi. & E a \\
A gdy przyjdzie zapomnieć i w pamięci to zatrzeć? & d a \\
Lepiej milczeć przytomnie i patrzeć. & E a \\[\zwrotkaspace]

\refrenspace Czy te oczy mogą kłamać\ldots \\

\end{piosenka}