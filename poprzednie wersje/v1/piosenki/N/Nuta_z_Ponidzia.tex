\begin{piosenka}{Nuta z Ponidzia -- WGB}

\akordy{a F E a F E}\\[\zwrotkaspace]

Polami, polami, po miedzach, po miedzach, & a F G C$^{7+}$ \\
Po błocku skisłym, w mgłę i wiatr & d$^7$ G C$^{7+}$ \\
Nie za szybko, kroki drobiąc & h$^7$ E$^7$ \\
Idzie wiosna, idzie nam & a G$^6$ F$^{7+}$ G \\
Idzie wiosna, idzie\ldots & a G e E \\[\zwrotkaspace]

\akordy{a F E a F E}\\[\zwrotkaspace]

Rozłożyła wiosna spódnicę zieloną & a F G C$^{7+}$ \\
Przykryła błota bury błam & d$^7$ G C$^{7+}$ \\
Pachnie ziemia ciałem młodym & h$^7$ E$^7$ \\
Póki wiosna, póki trwa & a G$^6$ F$^{7+}$ G \\
Póki wiosna, póki trwa & a G e E \\[\zwrotkaspace]

\akordy{a F E a F E}\\[\zwrotkaspace]

Rozpuściła wiosna warkocze kwieciste & a F G C$^{7+}$ \\
Zbarwniały łąki niczym kram & d$^7$ G C$^{7+}$ \\
Będzie odpust pod Wiślicą & h$^7$ E$^7$ \\
Póki wiosna, póki trwa & a G$^6$ F$^{7+}$ G \\
Póki wiosna, póki trwa & a G e E \\[\zwrotkaspace]

\akordy{a F E a F E}\\[\zwrotkaspace]

Ponidzie wiosenne, Ponidzie leniwe & a F G C$^{7+}$ \\
Prężysz się, jak do słońca kot & d$^7$ G C$^{7+}$ \\
Rozciągnięte na tych polach & h$^7$ E$^7$ \\
Lichych lasach, pstrych łozinach & h$^7$ E$^7$ \\
Skałkach słońcem rozognionych & h$^7$ E$^7$ \\
Nidą w łąkach roziskrzoną & h$^7$ E$^7$ \\
Na Ponidziu wiosna trwa & a G$^6$ F$^{7+}$ G \\
Na Ponidziu wiosna trwa & a G$^6$ F$^{7+}$ G \\
Na Ponidziu\ldots & a G F a \\

\end{piosenka}
\\
\chord{t}{x,p3,p2,n,n,n}{C$^{7+}$}
\chord{t}{x,x,n,p2,p1,p1}{d$^7$}
\chord{t}{p3,p2,n,n,n,n}{G$^6$}
\chord{t}{x,p3,p3,p2,p1,n}{F$^{7+}$}