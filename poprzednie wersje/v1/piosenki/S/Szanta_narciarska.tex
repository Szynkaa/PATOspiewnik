{\small \begin{piosenka}{Szanta narciarska -- Artur Andrus}
Nazywali go marynarz, & d C d \\
Bo opaskę miał na oku. & F G A \\
Na każdym stoku dziewczyna, & B F \\
Dziewczyna na każdym stoku. & F E d \\
Pochodzi spod Poznania, & d C d \\
Podobno umie wróżyć z kart. & F G A \\
Panny rwie na wiązania, & B F \\
Mężatki -- na długość nart. & F E d \\[1mm]

\refrenspace Caryco mokrego śniegu & A d \\
\refrenspace Ratrakiem płynę do Ciebie pod prąd\ldots Hej! & A B \\
\refrenspace Dobrze, że stoisz na brzegu, & B F \\
\refrenspace Bo ja właśnie schodzę na ląd. & F E d \\[1mm]

Nigdy się nie lękał biedy & d C d \\
I się nie przejmował jutrem. & F G A \\
A jego ratrak był kiedyś & B F \\
Zwyczajnym rybackim kutrem. & F E d \\
I woził dorsze i śledzie, & d C d \\
Zimą i latem, okrągły rok. & F G A \\
Teraz jak nieraz przejedzie & B F \\
Rybami czuć cały stok. & F E d \\[1mm]

\refrenspace Caryco mokrego śniegu\ldots \\[1mm]

Wszyscy w porcie odetchnęli. & d C d \\
Zwiał nim się zakończył sezon. & F G A \\
Jeszcze się tam jak żagiel bieli & B F \\
Jego czarny kombinezon. & F E d \\
Odpłynął pod Ustrzyki & d C d \\
I przez kobiety wpadł w kłopoty. & F G A \\
Forsę z polowań na orczyki & B F \\
Przehulał na antybiotyk. & F E d \\[1mm]

\refrenspace Caryco mokrego śniegu\ldots \\[1mm]

Jeśli kiedyś go zobaczysz & d C d \\
Na ratraku w podłym świecie, & F G A \\
To powiedz mu, że w Karpaczu & B F \\
Czekają na niego dzieci. & F E d \\
I kiedy opuszcza statek, & d C d \\
Żeby się znowu oddać złu, & F G A \\
Każda z dwudziestu siedmiu matek & B F \\
Dzieciątku śpiewa do snu: & F E d \\[1mm]

\refrenspace Caryco mokrego śniegu\ldots \\[1mm]
\end{piosenka} }