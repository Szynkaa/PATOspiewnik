\begin{piosenka}{Grosza daj Wiedźminowi -- Marcin Franc}
\textit{kapodaster na I progu}\\[\zwrotkaspace]

Tę balladę wam śpiewa skromny bard & a D d \\
Co z Geraltem z Rivii wyruszył na szlak & e D e E \\
Diaboła spotkał tam, co szukał z nim zwady & a D \\
I z elfów hufcami urządzał biesiady & G D G E \\[\zwrotkaspace]

Pochwycili mnie podstępem, no bo jak?! & a D d \\
Zniszczyli mi lutnię, skopali jak psa! & G E a \\
Ciała nasze dźgał ten rogaty stwór, & a D d \\
Zapłakał nasz wiedźmin, ,,Mam dosyć już!'' & G G E \\[\zwrotkaspace]

\refrenspace Grosza daj Wiedźminowi, & a E C \\
\refrenspace Sakiewką potrząśnij $\|\times2$ ło, o, o! & D a \\
\refrenspace Grosza daj Wiedźminowi, & a E C \\
\refrenspace Sakiewką potrząśnij! & D E \\[\zwrotkaspace]

Lecz chwycił Biały Wilk za morderczy róg, & a D F \\
Co tylu już przed nim obalił był z nóg & G D G E \\
Elfy cisnął precz, aż na górski szczyt, & a D F \\
Daleko od ludzi, gdzie miejsce ich & G D G E \\[\zwrotkaspace]

Choć oberwał sam, zmiażdżył bestii kark, & a D F \\
Ten obrońca ludzkości, toastu jest wart & G E a \\
Oto moja pieśń, to wasz bohater jest, & a D F \\
On wrogów pokonał, nalejcie mu więc! & G \\[\zwrotkaspace]

\refrenspace Grosza daj Wiedźminowi, & a E C \\
\refrenspace Sakiewką potrząśnij $\|\times2$ ło, o, o! & D a \\
\refrenspace Grosza daj Wiedźminowi, & a E C \\
\refrenspace Obrońcy ludzkości! & D E \\
\end{piosenka}