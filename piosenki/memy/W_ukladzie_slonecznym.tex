{\small \begin{piosenka_dluga}[5mm]{W układzie słonecznym -- NutkoSfera}
\akordy{A E cis D $\| \times 2$}\\[\zwrotkaspace]

\refrenspace W układzie słonecznym wirują planety & A E fis D \\*
\refrenspace Wszystkie w innym tempie, okrążają słońce & A E cis D \\*
\refrenspace Różnią się kolorem, masą i rozmiarem & A E fis D \\*
\refrenspace Lata lecą one stale, kręcą się wytrwale & A E cis D \\[\zwrotkaspace]

\textit{MERKURY}\\*[1mm]

Merkury? To ja! Jestem pierwszy i najmniejszy & A E \\*
Moją powierzchnię pokrywają ogromne kratery & fis D \\*
Z własną pogodą nie umiem dojść do zgody & A E \\*
W dzień jest mi gorąco, w nocy bardzo marznę brrr & cis D \\[\zwrotkaspace]

\textit{WENUS}\\*[1mm]

Mam na imię Wenus, jestem druga w kolejności & A E \\*
U mnie nieustannie upał jest bezlitosny & fis D \\*
Wiruję wolno, nigdzie się nie śpieszę & A E \\*
I bardzo jasno świecę się na niebie & cis D \\[\zwrotkaspace]

\textit{ZIEMIA}\\*[1mm]

Tu ziemia, witam serdecznie, czuje się świetnie & A E \\*
Zamieszkują mnie roślinki, zwierzątka i ludzie & fis D \\*
Jestem trzecią planetą od słońca, moja atmosfera jest cudna & A E \\*
Mam dużo wody, więc śmiało wpadnij ochłodzić & cis D \\[\zwrotkaspace]

\textit{MARS}\\*[1mm]

Hej tu Mars planeta numer cztery & A E \\*
Bardzo się czerwienię, jestem cały zardzewiały & fis D \\*
Mam najwyższe góry i od groma pyłu & A E \\*
A! Nic nie widzę, znowu leci mi do oczu! & cis D \\[\zwrotkaspace]
 
\refrenspace W układzie słonecznym\ldots \\[\zwrotkaspace]
 
\textit{JOWISZ}\\*[1mm]

Joł, joł, Jowisz, największy olbrzym gazowy & C a E a \\*
Piąty od słońca, na pewno nie przeoczysz & C a E a \\*
Choć moja masa jest potężna niezwykle & C a E a \\*
To z wszystkich planet kręcę się najszybciej & C a E a \\[\zwrotkaspace]
 
\textit{SATURN}\\*[1mm]

Siemanko tutaj Saturn, też składam się z gazów & A G \\*
Po moich pierścieniach poznasz mnie od razu & C E \\*
Wirują w nich kamienie, pył i sporo lodu & A G \\*
Jestem szósty od słońca, w głowie to zakoduj & C E \\[\zwrotkaspace]

\textit{URAN}\\*[1mm]

Cześć, to ja, lodowy gigant o imieniu Uran & C a E a \\*
Na mnie jest najniższa temperatura & C a E a \\*
Słynę z bladego, błękitnego koloru & C a E a \\*
Jestem siódmą planetą i wiruje na boku & C a E a \\[\zwrotkaspace]
 
\textit{NEPTUN}\\*[1mm]

Na koniec ja swoje trzy grosze wtrącę & C a E a \\*
Nazywam się Neptun i najdłużej okrążam słońce & C a E a \\*
Tworzę huragany co powalą wszystko & C a E a \\*
Więc nie zbliżaj się zbytnio bo cię zdmuchnę jak piórko & C a E a a \\[\zwrotkaspace]

\refrenspace W układzie słonecznym wirują planety & A E fis D \\*
\refrenspace Wszystkie w innym tempie, okrążają słońce & A E cis D \\*
\refrenspace Różnią się kolorem, masą i rozmiarem & A E fis D \\*
\refrenspace Lata lecą one stale, kręcą się wytrwale & A E cis D \\[\zwrotkaspace]

\akordy{C a E a $\| \times 2$}\\
\end{piosenka_dluga}
\vspace{6mm}
\begin{flushleft}
Kosmiczne ciekawostki:
Czy wiecie że nie tylko Saturn ma pierścienie? Jowisz, Uran i Neptun także! Jednak to pierścienie Saturna są najlepiej widoczne z Ziemi.
\end{flushleft}}

