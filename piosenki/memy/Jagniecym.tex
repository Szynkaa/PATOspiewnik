\begin{piosenka}{Jagnięcym futerkiem wałek pokryty -- Adam z Lisińca}

\refrenspace Jagnięcym futerkiem wałek pokryty & d d$^4$ d C d \\
\refrenspace Na metalowym leży regale & d F a d \\
\refrenspace Trochę widoczny, choć trochę skryty & d d$^4$ d C d \\
\refrenspace Przywodzi na myśl tatrzańskie hale & d F a d \\[\zwrotkaspace]

Czyją to była owa owieczka & d F C d \\
Której futerkiem malujesz krużganek? & d F C d \\
Hasała po łąkach jak tancereczka & d F C d \\
A teraz po niej pozostał ten wałek\ldots & d F C d \\[\zwrotkaspace]

\refrenspace Jagnięcym futerkiem wałek pokryty & d F C d \\
\refrenspace Na metalowym leży regale & d F C d \\
\refrenspace Trochę widoczny, choć trochę skryty & d F C d \\
\refrenspace Przywodzi na myśl tatrzańskie hale & d F a d \\[\zwrotkaspace]

Owieczko pogodna, gdzie masz Pasterza? & d F C d \\
Czy nie wiesz, że wilk przerobi cię w wałek? & d F C d \\
Pasterz nakarmi, gdy będziesz głodna, & d F C d \\
Nie sprzeda nikomu -- On cię ocali. & d F a d \\[\zwrotkaspace]

Włoży w Twe usta słowa kwieciste, & d F C d \\
Przestaniesz beczeć tym głosem baranim. & d F C d \\
Zaśpiewasz wtedy pieśni wieczyste, & d F C d \\
Na chmurce odtańczysz niebiański balet. & d F a d \\[\zwrotkaspace]

\refrenspace Jagnięcym futerkiem wałek pokryty & d F C d \\
\refrenspace Na metalowym leży regale & d F C d \\
\refrenspace Trochę widoczny, choć trochę skryty & d F C d \\
\refrenspace Przywodzi na myśl tatrzańskie hale & d F a d \\[\zwrotkaspace]


\end{piosenka}
\begin{figure}[H]
	\centering 
	\includegraphics[width=0.45\textwidth]{piosenki/memy/jagniecym_futerkiem.jpg}	
\end{figure}
