\begin{piosenka}{Opowieści Złotej Fali -- Atlantyda}

\textit{kapodaster na II progu}\\[\zwrotkaspace]

Fala, wielka fala burty nasze rwie, & a \\
Pędzi nas na grzbiecie, niesie diabli wiedzą gdzie. & a \\
Błysk rozdziera czarne niebo, leje deszcz & d e \\
I sztormowe morze jest. & a \\[\zwrotkaspace]

Tak, jak brzytwą, żagle rozpruł dziki wiatr, & a \\
Wędrujemy z falą, a do domu drogi szmat. & a \\
Z każdej strony masy wody leją się, & d e \\
Z boków fale, z góry deszcz. & a \\[\zwrotkaspace]


\refrenspace Wystrzępione wszystkie żagle aż po drzewce rej, & C G \\
\refrenspace A ładownia przypomina dzban. & d a \\
\refrenspace Bierze wodę wciąż burtami i napełnia się, & C G \\
\refrenspace Jeszcze kilka kropli wpadnie i będziemy wszyscy na dnie. & d E$^7$ \\[\zwrotkaspace]

Fala, wielka fala rozwaliła ster, & a \\
Łajba, jak tancerka w karnawale, tańczyć chce. & a \\
Cieśla wielkim młotem wbija twardy szpunt, & d e \\
Do cieknących szpar i dziur. & a \\[\zwrotkaspace]

Sztorm to wielka siła, sztorm to morza gniew, & a \\
Stary wciąż na deku ryczy jak zraniony lew. & a \\
Pompy duszą się od wody, gubią rytm, & d e \\
Ludzie też nie mają sił. & a \\[\zwrotkaspace]

\refrenspace Wystrzępione wszystkie żagle\ldots\\[\zwrotkaspace]

Fala, wielka fala i marynarski los, & a \\
Stary już od tego krzyku całkiem stracił głos. & a \\
Tylko ręce pokazują, że to ląd, & d e \\
Że już blisko jest nasz dom. & a \\[\zwrotkaspace]

Pędzi złota fala, którą dobrze znam, & a \\
Znika w gardle i poprawia humor wszystkim nam. & a \\
Teraz pusta stoi szklanka, no więc cóż, & d e \\
Opowieści koniec już. & a \\[\zwrotkaspace]

\refrenspace Wystrzępione wszystkie żagle\ldots\\[\zwrotkaspace]

\end{piosenka}