\begin{piosenka_dluga}{Alfabet bosmański -- Stare Dzwony}
A -- jak Atlantyk, co trzeba go przejść, & a G a \\*
B -- jak burta, a burty są dwie, & a F E \\*
C -- jak cuma -- ,,Hej, wybierz i obłóż ją!'' & a G \\*
D -- jak dryfkotwa i dirki i dno. & a G \\[\zwrotkaspace]

\refrenspace Hey derry, hay derry, hey derry down! & a F \\*
\refrenspace Masz kłopot -- weź łyka -- i kłopot już znika, & C G \\*
\refrenspace Po rumie, w tym szumie, nie może być źle! & a F \\*
\refrenspace Po rumie każdemu na morze się chce & C G \\[\zwrotkaspace]

E -- jak Eol, co sprzyja dziś nam, & a G a \\*
F -- jak fały -- ,,Hej, wybierać fał!'' & a F E \\*
G -- jak gejtawy, gordingi i grot, & a G \\*
H -- jest jak handszpak -- trza mocno pchać go. & a G \\[\zwrotkaspace]

%\refrenspace Hey derry, hay derry\ldots \\[\zwrotkaspace]

I -- to sygnał: ,,W lewo zmieniam mój kurs.'' & a G a \\*
J -- jak juzing -- trzeba żagle szyć znów, & a F E \\*
K -- jak kubryk, skąd leci nasz śpiew, & a G \\*
L -- jak latarnia, co błyski nam śle. & a G \\[\zwrotkaspace]

%\refrenspace Hey derry, hay derry\ldots \\[\zwrotkaspace]

Ł -- jak łańcuch -- ,,Hej, wybieraj go!'' & a G a \\*
M -- jak marsel, poniżej jest grot, & a F E \\*
N -- jak naktuz i nagiel, i nok, & a G \\*
O -- jak obijacz -- ,,Za burtę więc go!'' & a G \\[\zwrotkaspace]

%\refrenspace Hey derry, hay derry\ldots \\[\zwrotkaspace]

P -- jak pompy -- już znamy ten ruch, & a G a \\*
Q -- oznacza: ,,Mój statek jest zdrów.'' & a F E \\*
R -- jak reje, gdzie łazimy co dzień, & a G \\*
S -- tak jak stenga i saling, i ster. & a G \\[\zwrotkaspace]

%\refrenspace Hey derry, hay derry\ldots \\[\zwrotkaspace]

T -- jak trapy, co wiodą na ląd, & a G a \\*
U -- jak Uznam, gdzie znam każdy kąt, & a F E \\*
W -- jak wimpel -- prostuje go wiatr, & a G \\*
Z -- tak jak zejman, co płynie przez świat. & a G \\[\zwrotkaspace]

%\refrenspace Hey derry, hay derry\ldots \\[\zwrotkaspace]
\end{piosenka_dluga}