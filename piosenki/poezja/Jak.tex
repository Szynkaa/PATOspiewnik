\begin{piosenka}{Jak -- SDM}
Jak po nocnym niebie sunące & A E \\
Białe obłoki nad lasem & D A \\
Jak na szyi wędrowca apaszka szamotana wiatrem & h D A A$^4$\\[\zwrotkaspace]

Jak wyciągnięte tam powyżej & A E \\
Gwiaździste ramiona wasze & D A \\
A tu są nasze, a tu są nasze & h D A A$^4$ \\[\zwrotkaspace]

Jak suchy szloch w tę dżdżystą noc & A E \\
Jak winny-li-niewinny sumienia wyrzut & D A \\
Że się żyje, gdy umarło tylu, tylu, tylu & h D A A$^4$ \\[\zwrotkaspace]

Jak suchy szloch w tę dżdżystą noc & A E \\
Jak lizać rany celnie zadane & D A \\
Jak lepić serce w proch potrzaskane & h D A A$^4$ \\[\zwrotkaspace]

Jak suchy szloch w tę dżdżystą noc & A E \\
Pudowy kamień, pudowy kamień & D A \\
Ja na nim stanę, on na mnie stanie & h D \\
On na mnie stanie, spod niego wstanę & A A$^4$ \\[\zwrotkaspace]

Jak suchy szloch w tę dżdżystą noc & A E \\
Jak złota kula nad wodami & D A \\
Jak świt pod spuchniętymi powiekami & h D A A$^4$ \\[\zwrotkaspace]

Jak zorze miłe, śliczne polany & A E \\
Jak słońca pierś, jak garb swój nieść & D A \\
Jak do was, siostry mgławicowe & h D \\
Ten zawodzący śpiew & A A$^4$ \\[\zwrotkaspace]

Jak biec do końca, potem odpoczniesz & A E \\
Potem odpoczniesz, cudne manowce & D A \\
Cudne manowce, cudne, cudne manowce $\Vert\ \times \infty$ & h D A A$^4$ \\
\end{piosenka}\\
\chord{t}{x,n,p2,p2,p3,n}{A$^4$}
