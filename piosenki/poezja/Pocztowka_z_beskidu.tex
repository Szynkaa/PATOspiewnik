\begin{piosenka}{Pocztówka z Beskidu -- Wołosatki}

Po Beskidzie błądzi jesień & G D \\
Wypłakuje deszczu łzy. & e h \\
Na zgarbionych plecach niesie & C G \\
Worek siwej mgły. & a$^7$ D \\
Pastelowe cienie kładzie & G D \\
Zdobiąc rozczochrany las. & e h \\
Nocą rwie w brzemiennym sadzie & C G \\
Grona słodkich gwiazd, złotych gwiazd. & a$^7$ D G G$^7$ \\[\zwrotkaspace]

\refrenspace Jesienią góry są najszczersze, & C D G C \\
\refrenspace Żurawim kluczem otwierają drzwi. & C D G G$^7$ \\
\refrenspace Jesienią smutne piszę wiersze,	& C D G C \\
\refrenspace Smutne piosenki śpiewam Ci. & C D G \\[\zwrotkaspace]

Po Beskidzie błądzą ludzie,  & G D \\
Kare konie w chmurach rżą. & e h \\
Święci pańscy zamiast w niebie & C G \\
Po kapliczkach śpią. & a$^7$ D \\
Kowal w kuźni klepie biedę, & G D \\
Czarci wydeptują trakt. & e h \\
W pustej cerkwi co niedzielę & C G \\
Rzewnie śpiewa wiatr, pobożny wiatr. & a$^7$ D G G$^7$ \\[\zwrotkaspace]

\refrenspace Jesienią góry są najszczersze\ldots \\

\end{piosenka}