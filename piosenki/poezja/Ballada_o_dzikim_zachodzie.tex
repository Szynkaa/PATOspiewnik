\begin{piosenka_dluga}{Ballada o dzikim zachodzie -- Wojciech Młynarski} 

Potwierdzają to setne przykłady & D G D \\*
Że westerny wciąż jeszcze są w modzie & h A D \\*
Wysłuchajcie więc państwo ballady & D G D \\*
O tak zwanym najdzikszym zachodzie & h A D \\*
Miasto było tam jakich tysiące & G D \\*
Wokół preria i skały naprzeciw & G D \\*
Jak gdzie indziej świeciło tam słońce & D G D \\*
Marli starcy, rodziły się dzieci & h A D \\[\zwrotkaspace]

\refrenspace I tym tylko od innych różni się ta ballada & G A G D \\*
\refrenspace Że w tym mieście gdzieś na prerii krańcach & G D \\*
\refrenspace Na jednego mieszkańca jeden szeryf przypadał & G A G D \\*
\refrenspace Jeden szeryf na jednego mieszkańca & h A D \\[\zwrotkaspace]

Konsekwencje ten fakt miał ogromne & D G D \\*
Bo nikt w mieście za spluwę nie chwytał & h A D \\*
I od dawna już każdy zapomniał & D G D \\*
Jak wygląda prawdziwy bandyta & h A D \\*
Choć finanse poniekąd leżały & G D \\*
Gospodarka i przemysł był na nic & G D \\*
Ale każdy, czy duży, czy mały & D G D \\*
Czuł się za to bezpieczny bez granic & h A D \\[\zwrotkaspace]

\refrenspace Bo tym tylko od innych różni się ta ballada & G A G D \\*
\refrenspace Że w tym mieście gdzieś na prerii krańcach & G D \\* 
\refrenspace Na jednego mieszkańca jeden szeryf przypadał & G A G D \\*
\refrenspace Jeden szeryf na jednego mieszkańca & h A D \\[\zwrotkaspace]

Jeśli państwa historia ta nudzi & D G D \\*
To pocieszcie się tym, że nareszcie & h A D \\*
Którejś nocy krzyk ludzi obudził & D G D \\*
Bank rozbity, bandyci są w mieście & h A D \\*
Dobrzy ludzie, na próżno wołacie & G D \\*
Nikt nie wstanie, za spluwę nie chwyci & G D \\*
Skoro każdy świadomość zatracił & D G D \\*
Czym się różnią od ludzi bandyci ­& h A D \\[\zwrotkaspace]

\refrenspace A tym tylko od innych różni się ta ballada & G A G D \\*
\refrenspace Że w tym mieście gdzieś na prerii krańcach & G D \\*
\refrenspace Na każdego człowieka nagle strach upadł blady & G A G D \\*
\refrenspace Od szeryfa do zwykłego mieszkańca & h A D \\[\zwrotkaspace]

Potwierdzają to setne przykłady & D G D \\*
Że westerny wciąż jeszcze są w modzie & h A D \\*
Wysłuchaliście państwo ballady & D G D \\*
O tak zwanym najdzikszym zachodzie & h A D \\*
Miasto było tam jakich tysiące & G D \\*
Ludzkie w nim krzyżowały się drogi & G D \\*
Lecz nie wszystkim świeciło tam słońce & D G D \\*
Bo bandyci krążyli bez trwogi & h A D \\[\zwrotkaspace]

\refrenspace Wyciągnijmy więc morał w tej balladzie ukryty & G A G D \\*
\refrenspace Gdy nie grozi nam żadne rififi & G D \\* 
\refrenspace Że czasami najtrudniej rozpoznać bandytę & G A G D \\*
\refrenspace Gdy dokoła są sami szeryfi & h A D \\[\zwrotkaspace]

\end{piosenka_dluga}