\begin{piosenka}{Sen Katarzyny II -- Jacek Kaczmarski}
Na smyczy trzymam filozofów Europy & G D G \\*
Podparłam armią marmurowe Piotra stropy & G D e \\*
Mam psy, sokoły, konie, kocham łów szalenie & C D e \\*
A wokół same zające i jelenie & C D G \\*
Pałace stawiam, głowy ścinam & Fis h \\*
Kiedy mi przyjdzie na to chęć & Fis G D \\*
Mam biografów, portrecistów & C D e \\*
I jeszcze jedno pragnę mieć & C D G \\[\zwrotkaspace]

\refrenspace Stój, Katarzyno! Koronę Carów & e h e h \\*
\refrenspace Sen, taki jak ten, może ci z głowy zdjąć! & e a C D G \\[\zwrotkaspace]

Kobietą jestem ponad miarę swoich czasów & G D G \\*
Nie bawią mnie umizgi bladych lowelasów & G D e \\*
Ich miękkich palców dotyk budzi obrzydzenie & C D e \\*
Już wolę łowić zające i jelenie & C D G \\*
Ze wstydu potem ten i ów & Fis h \\*
Rzekł o mnie: Niewyżyta Niemra & Fis G D \\*
I pod batogiem nago biegł & C D e \\*
Po śniegu dookoła Kremla & C D G \\[\zwrotkaspace]

\refrenspace Stój, Katarzyno\ldots \\[\zwrotkaspace]

Kochanka trzeba mi takiego jak imperium & G D G \\*
Co by mnie brał tak jak ja daję - całą pełnią & G D e \\*
Co by i władcy i poddańca był wcieleniem & C D e \\*
I mi zastąpił zające i jelenie & C D G \\*
Co by rozumiał tak jak ja & Fis h \\*
Ten głupi dwór rozdanych ról & Fis G D \\*
I pośród pochylonych głów & C D e \\*
Dawał mi rozkosz albo ból & C D G \\[\zwrotkaspace]

\refrenspace Stój, Katarzyno! Koronę Carów & e h e h \\*
\refrenspace Sen, taki jak ten, może ci z głowy zdjąć! & e a C D G \\*
\refrenspace Gdyby się taki kochanek kiedyś znalazł & e h e h \\*
\refrenspace Wiem! Sama wiem! Kazałabym go ściąć! & e a C D G \\*
\end{piosenka}
