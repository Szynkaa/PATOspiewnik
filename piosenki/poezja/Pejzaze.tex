\begin{piosenka}[2mm]{Pejzaże harasymowiczowskie -- WGB}
Kiedy stałem w przedświcie, a Synaj & G D \\
Prawdę głosił przez trąby wiatru & C e \\
Zasmreczyły się chmury igliwiem & G D \\
Bure świerki o górach wsparte \textit{/Bure świerki/} & e C D \\[\zwrotkaspace]

I na niebie byłem ja jeden & G D \\
Plotąc pieśni w warkocze bukowe & C e \\
I schodziłem na ziemię za kwestą & G D \\
Przez skrzydlącą się Bramę Lackowej \textit{/Przez Lackową/} & e C D \\[\zwrotkaspace]

\refrenspace I był Beskid, i były słowa & G C G \\
\refrenspace Zanurzone po pępki w cerkwi baniach & G C D \\
\refrenspace Rozłożyście złotych & D \\
\refrenspace Smagających się wiatrem do krwi & C D G \\[\zwrotkaspace]

Moje myśli biegały końmi & G D \\
Po niebieskich, mokrych połoninach & C e G \\
I modliłem się, złożywszy dłonie \textit{/I modliłem się wtedy/} & G D \\
Do gór, do Madonny brunatnolicej \textit{/Do Madonny/} & e C D \\[\zwrotkaspace]

A gdy serce kroplami tęsknoty & G D \\
Jęło spadać na góry sine & C e \\
Czarodziejskim kwiatem paproci \textit{/Czarodziejskim kwiatem/} & G D \\
Rozgwieździła się Bukowina \textit{/Bukowina/} & e C D \\[\zwrotkaspace]

\refrenspace I był Beskid\ldots \\[\zwrotkaspace]
\end{piosenka}
