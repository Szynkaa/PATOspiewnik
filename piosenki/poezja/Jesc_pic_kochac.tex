\begin{piosenka}{Jeść, pić, kochać -- Pod Budą}
Kiedy poranną sączę kawę & C G C \\
Rozgrzany po niedawnym śnie & C G G$^7$ \\
Przeglądam pierwsze strony gazet & C G C \\
I mówiąc szczerze boję się & d G G$^7$ \\
Wokół ruiny i pożogi & C G C \\
Płyną powodzie spada śnieg & C G \\
I wszędzie twarze pełne trwogi & d G \\
Bo przyszedł już kolejny wiek & d G \\[\zwrotkaspace]

\refrenspace Więc ci dziękuję losie, choćby tylko za to, & C G C \\ 
\refrenspace Że nie musiałem się urodzić pod wulkanem, & F d G \\
\refrenspace Że średni u nas klimat i przeciętne lato, & C C$^7$ F \\
\refrenspace Ale dzieciaki są przeważnie roześmiane. & d G \\
\refrenspace I chociaż czasem przyfruwają szare dni, & C G C \\ 
\refrenspace A przez mój ogród nie chce płynąć żyła złota, & F d G \\
\refrenspace To przecież zawsze mogłem robić rzeczy trzy: & C C$^7$ F \\
\refrenspace Jeść, pić, kochać & C d C G \\[\zwrotkaspace]

Kiedy poranną sączę kawę & C G C \\
I topię w niej niedawny sen & C G G$^7$ \\
Przeglądam pierwsze strony gazet & C G C \\
To jedno wiem, naprawdę wiem & d G G$^7$ \\
Gdy dookoła puste słowa & C G C \\
I nowa bitwa wciąż u drzwi & C G \\
To trzeba umieć uszanować & d G \\
Tę jedną chwilę, która lśni & d G \\[\zwrotkaspace]

\refrenspace Więc ci dziękuję losie\ldots \\
\end{piosenka}