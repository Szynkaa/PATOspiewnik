{\small \begin{piosenka}{Major Ponury}

Mgła wschodzi z lasu, Panie Majorze. & e D \\
Wiatr się zrywa z chaszczy jak ptak. & C G D \\
Już się szkopy nie tułają po borze, & e h \\
Niejednego przez nas trafił szlag. & C D \\
Jutro do wsi pewnie zajdziemy. & e D \\
Pies nie szczeknie -- przecież my swoi & C G D \\
U mej matuli cokolwiek zjemy,  & e h \\
Potem śpiewaniem do snu ukoi\ldots & C D \\[\zwrotkaspace]

\refrenspace I dobrze odpoczniem nim odejdziem w góry & C D \\
\refrenspace Lecz co Pan Major taki ponury? & e D (e D e) \\
\refrenspace $\| \times 2$ \\[\zwrotkaspace]

Do diabła ze śmiercią, Panie Majorze! & e D \\
Pan szedł z nią razem w 39-tym & C G D \\
Potem trza było z wojskiem się łączyć & e h \\
I miecze ostrzyć daleko za morzem. & C D \\
Myśmy czekali, bo wodza brakło. & e D \\
Lichy to zwierz, co walczy bez oka. & C G D \\
Wieści przysłali i słowo sie rzekło & e h \\
Biały orzeł z góry spikował\ldots & C D \\[\zwrotkaspace]

\refrenspace I w piersi wroga wbił swe pazury,  & C D \\
\refrenspace Lecz co Pan Major taki ponury? & e D (e D e) \\
\refrenspace $\| \times 2$ \\[\zwrotkaspace]

To nie był taki zwyczajny bój. & e D \\
Lufa się zgrzała jak klucze od piekła. & C G D \\
Mocno się wrzynał w kieszeni nabój & e h \\
I każda chwila się w wieczność wlekła. & C D \\
Strasznie Pan dostał, Panie Majorze. & e D \\
Jak mi Bóg miły nie mogło być gorzej. & C G D \\
Krew się przelała przez głębokie rany, & e h \\
Archanioł Michał otworzył bramy\ldots & C D \\[\zwrotkaspace]

\refrenspace Pozdrówcie ode mnie Świętokrzyskie Góry,  & C D \\
\refrenspace Szepnął i skonał Major Ponury & e D (e D e) \\
\refrenspace $\| \times 2$ \\[\zwrotkaspace]

\refrenspace Skonał i odszedł odnaleźć swe góry,  & C D \\
\refrenspace Z serca bohater Major Ponury\ldots & e D (e D e) \\
\refrenspace $\| \times 2$ \\[\zwrotkaspace]

\end{piosenka} }