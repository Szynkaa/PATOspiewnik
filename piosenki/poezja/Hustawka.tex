{\normalsize \begin{piosenka}[2mm]{Huśtawka -- Kwiat Jabłoni}

A czy przyroda w kolebkach myślała kiedyś dokładnie & G D \\
Na co jej wielkie mamuty? Ani wygląda to ładnie & e C \\
Ani z nich skóra na buty & G D \\
Nie ma co pytać koledzy, robiła i tak jej wyszło & G D \\
Nikt nie wymyślał specjalnie tego, w czym żyć nam przyszło & e C \\
uprzedzam o tym lojalnie & G D \\[1.8mm]

\refrenspace Jeden jest rytm, jeden rytm, jeden jest węgiel i tlen & G D \\
\refrenspace Zwykłą losu koleją, praca, posiłek i sen & e C \\[1.8mm]

\refrenspace Jeden przypada na dzień, świt jeden i jeden zmrok & G D \\
\refrenspace Pierwsi się łudzą nadzieją, a drudzy równają krok & e C \\
\refrenspace $\| \times 2$ \\[1.8mm]

Nie skacz tak zaraz na szyny, jeszcze nie o tę grasz stawkę & G D \\
W wesołym miasteczku dziewczyny chcą z tobą iść na huśtawkę & e C\\
Lepiej ci będzie z nimi & G D \\
Pachnie tak mocno siano, kwiaty się gną od motyli & G D \\
Jeździ słońce po niebie, światło ucieka, ślad myli & e C \\
Miasteczko czeka na ciebie & G D \\[1.8mm]

\refrenspace Jeden jest rytm, jeden rytm, ważny jest wydech i wdech & G D \\
\refrenspace Nasyć się równym oddechem, nasyć się dzisiaj za trzech & e C \\[1.8mm]

\refrenspace Raz tylko dany ci czas, ani on twój ani czyj & G D \\
\refrenspace Z czasem się wszystko ustoi, żyj na huśtawce żyj & e C \\
\refrenspace $\| \times 2$ \\[1.8mm]

\refrenspace Jeden jest rytm, jeden rytm, jeden jest węgiel i tlen & G D \\
\refrenspace Zwykłą losu koleją, praca, posiłek i sen & e C \\[1.8mm]

\refrenspace Jeden przypada na dzień, świt jeden i jeden zmrok & G D \\
\refrenspace Pierwsi się łudzą nadzieją, a drudzy równają krok & e C \\[1.8mm]

\refrenspace Jeden jest rytm, jeden rytm, ważny jest wydech i wdech & G D \\
\refrenspace Nasyć się równym oddechem, nasyć się dzisiaj za trzech & e C \\[1.8mm]

\refrenspace Raz tylko dany ci czas, ani on twój ani czyj & G D \\
\refrenspace Z czasem się wszystko ustoi, żyj na huśtawce żyj & e C \\[1.8mm]

\refrenspace Żyj na huśtawce żyj, ani on twój ani czyj & G D \\
\refrenspace Ani czyj ani czyj, z czasem wszystko się ustoi & e C \\
\end{piosenka}}