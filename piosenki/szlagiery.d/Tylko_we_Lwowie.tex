\begin{piosenka}{Tylko we Lwowie -- Włóczęgi}
	
Niech inni se jadą gdzie mogą, gdzie chcą & f b \\
do Wiednia, Paryża, Londynu. & C f \\
A ja się ze Lwowa nie ruszę za próg, & f b \\
za skarby ta skarz mnie Bóg & G C \\[\zwrotkaspace]

\refrenspace Bo gdzie jeszcze ludziom tak dobrze jak tu? & F C \\
\refrenspace Tylko we Lwowie! & C C$^7$ F \\
\refrenspace Gdzie śpiewem cię tulą i budzą ze snu? & F C \\
\refrenspace Tylko we Lwowie! & C C$^7$ F \\[\zwrotkaspace]

\refrenspace I bogacz, i dziad tu są za pan brat & B F \\
\refrenspace I każdy ma uśmiech na twarzy. & C F \\
\refrenspace A panny to ma słodziutkie ten gród, & B F \\
\refrenspace Jak sok, czekolada i miód\ldots & G C \\[\zwrotkaspace]

\refrenspace I gdybym się kiedyś urodzić miał znów, & F C \\
\refrenspace To tylko we Lwowie! & C C$^7$ F \\
\refrenspace Bo szkoda gadania, bo co chcesz, to mów  & F C \\
\refrenspace Nie ma jak Lwów! & C C$^7$ F \\[\zwrotkaspace]

\akordy{f b C}\\[\zwrotkaspace]

Możliwe że dużo ładniejszych jest miast & f b \\
lecz Lwów to jest jeden na świecie  & C f \\
i z niego wyjechać to gdzież ja bym mógł & f b \\
ta mamciu ta skarz mnie Bóg & G C \\[\zwrotkaspace]

\refrenspace Bo gdzie jeszcze ludziom tak dobrze jak tu\ldots \\[\zwrotkaspace]
	
\end{piosenka}	