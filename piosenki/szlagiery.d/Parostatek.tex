\begin{piosenka}{Parostatek -- Krzysztof Krawczyk}

\akordy{F G$^7$ C} \\[\zwrotkaspace]

W starym albumie u mego dziadka & F \\
Jest takie zdjęcie, istny cud & D$^7$ \\
Płynący w falach, wśród mewek stadka & G C \\
Statek na parę sprzed lat stu & F g C \\[\zwrotkaspace]

Tłum marynarzy pokład mu zdobi & F \\
Słońce na górze pięknie lśni & D$^7$ \\
Dobry fotograf to zdjęcie zrobił & G C \\
Wszystko jak żywe, aż się ckni & F g C \\[\zwrotkaspace]

\refrenspace Parostatkiem piękny rejs, & F D$^7$ \\
\refrenspace Statkiem na parę piękny rejs & D$^7$ G$^7$ C F\\
\refrenspace Przy wtórze klątw bosmana, & E$^7$ F$^7$\\
\refrenspace Głośnych krzyków aż od rana & G$^7$\\
\refrenspace Tak śpiewnie dusza łka & C$^7$ F\\[\zwrotkaspace]

\refrenspace Kąpielowy kostium włóż & F D$^7$ \\
\refrenspace I na pokładzie ciało złóż & D$^7$ G$^7$ C F\\
\refrenspace Bo tutaj szum maszyny, & E$^7$ F$^7$\\
\refrenspace Bo tutaj głosem dziewczyny & G$^7$\\
\refrenspace Tak cudnie śruba gra & C$^7$ F\\[\zwrotkaspace]

\akordy{F G$^7$ C} \\[\zwrotkaspace]

Dziadek bosmanem był na tym statku & F \\
Wśród majtków wzbudzał wiecznie strach & D$^7$ \\
Krzyczał, az drżały na brzegach kwiatki & G C \\
Cała załoga stała w łzach & F g C \\[\zwrotkaspace]

Lecz kiedy dziadek fajkę zapalił & F \\
Tytoń mu zaczął płuca grzać & D$^7$ \\
Dziadek coś nucił, tytoń się palił & G C \\
Marzył, by wieki mógł mu trwać & F g C \\[\zwrotkaspace]

\refrenspace Parostatkiem piękny rejs\ldots \\
	
	
\end{piosenka}	