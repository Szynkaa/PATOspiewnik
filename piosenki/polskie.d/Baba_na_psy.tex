\begin{piosenka}{Baba na psy -- Artur Andrus}

Tak ogólnie to był słaby, chorowity i niewielki & d g \\
Ale straszny pies na baby, zwłaszcza na XL-ki & C F A \\
Kiedy wreszcie się ożenił, mówił do niej per ,,kobieto'' & d g \\
Nikt jej nie zna po imieniu, ale wszyscy wiedzą, że to & C A \\[\zwrotkaspace]

\refrenspace Baba na psy, baba na psy & d g \\ 
\refrenspace Ten świat cały psu na budę & C F A \\
\refrenspace Baba na psy, baba na psy & d g \\ 
\refrenspace Zwłaszcza na rude & A d \\[\zwrotkaspace]

Życie tak ją nauczyło --- rudy pies czy ruda suka & d g \\
Wie co to prawdziwa miłość i cię nie oszuka & C F A \\
Facet kosę wbije w pierś ci, choćby był rodzonym bratem & d g \\
Taki, z sierścią czy bez sierści, facet wredny jest a zatem & C A \\[\zwrotkaspace]

\refrenspace Baba na psy, baba na psy\ldots \\[\zwrotkaspace]

Rudy pies ma więcej błysku i polotu niż mężczyzna & d g \\
,,Głos Wybrzeża'' nosi w pysku, choć to Lubelszczyzna & C F A \\
Nigdy w życiu nie miał kaca, nie tłumaczy się kolegom, & d g \\
Że za chwilę musi wracać, bo tam w domu czeka jego & C A \\[\zwrotkaspace]

\refrenspace Baba na psy, baba na psy\ldots \\[\zwrotkaspace]

Kiedyś, ludzie, uwierzycie w bezgraniczną moc miłości & d g \\
Pies na babę spędzi życie z kobietą przy kości & C F A \\
A jak odejść będzie trzeba, w jakąś wiosnę albo lato & d g \\
Wezmą ich do psiego nieba --- bo zapracowała na to! & C A \\[\zwrotkaspace]

\refrenspace Baba na psy, baba na psy\ldots \\[\zwrotkaspace]

\end{piosenka}