\begin{piosenka}{Rapapara -- Łydka Grubasa}
On był samotny, jej było źle & a \\
Gdzieś w internecie poznali się & F \\
On się zakochał ze samych zdjęć & C \\
Bo tam rusałka, dziewczę na pięć & G \\
Szczęka mu spadła aż pod sam stół & a \\
Dał jej komentarz dziesięć i pół & F \\
A kiedy w końcu spotkali się & C \\
On jej nie poznał dlatego, że\ldots & G \\[1.3mm]

\refrenspace Rapapara, rapapara miała ryja jak kopara $\|\times2$ & a F C G \\[1.3mm]

On chciał zakochać się z całych sił & a \\
Lecz ciągle widział ten wielki ryj & F \\
W łóżku i w pracy, noce i dnie & C \\
Z hipopotamem kojarzył się & G \\[1.3mm]

\refrenspace Rapapara, rapapara miała ryja jak kopara $\|\times2$ & a F C G \\[1.3mm]

Aż w końcu przyszedł zimowy czas & a \\
Śniegu nasypało aż po pas & F \\
Gdy on do pracy wyruszyć chciał & C \\
Ujrzał, że w śniegu ugrzązł mu star & G \\
Płacząc przeklinał parszywy los & a \\
Wtedy: ,,pomogę'' -- usłyszał głos & F \\
I kiedy w starze zarzucał bieg & C \\
To ona ryjem spychała śnieg & G \\[1.3mm]

\refrenspace Rapapara, rapapara i tym ryjem jak kopara & a F C G \\
\refrenspace Rapapara, rapapara odkopała chłopu stara & a F C G \\[1.3mm]

Więc przyznaj się teraz, przyznaj się sam & a F \\
Śmiałeś się z ryja, śmiałeś jak cham & C G \\
I brałeś do ręki sękaty kij & a F \\
I plułeś, i szczułeś ten wielki ryj & C G \\
Lecz karty rozdaje parszywy los & a F \\
I ryj bywa cenny jak złota stos & C G \\
A więc nie śmiejcie się z cudzych wad & a F \\
Bo one mogą zbawić wasz świat & C G \\[1.3mm]

\refrenspace Rapapara, rapapara nawet morda jak kopara & a F C G \\
\refrenspace Rapapara, rapapara zasługuje na browara! & a F C G \\
\refrenspace $\|\times2$
\end{piosenka}