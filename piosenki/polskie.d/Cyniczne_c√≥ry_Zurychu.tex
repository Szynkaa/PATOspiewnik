\begin{piosenka_dluga}{Cyniczne córy Zurychu -- Artur Andrus}

Ma stary tata Turek sześć zaradnych córek: \\*
Ajsze, Baszak, Fatma, Dżanan, Burczu i Raszida. \\*
Gdy się na córki złości, to w tej kolejności: \\*
Ajsze, Baszak, Fatma, Dżanan, Burczu i Raszida. \\*
Którejś nocy wyszły z domu i uciekły po kryjomu \\*
Ajsze, Baszak, Fatma, Dżanan, Burczu i Raszida. \\*
Dostał tata wieści, z których wyszło, że są w mieście Zurych, \\*
Trafił tatę szlag \\*
I zakrzyknął tak: \\[\zwrotkaspace]
 
\refrenspace Cyniczne córy Zurychu  \\*
\refrenspace Potępiam was wszystkie w czambuł, \\*
\refrenspace Cyniczne córy Zurychu, \\*
\refrenspace Płacze za wami Stambuł. \\[\zwrotkaspace]

A już po latach paru wyszły za Szwajcarów \\*
Ajsze, Baszak, Fatma, Dżanan, Burczu i Raszida. \\*
I stało im się bliskie jezioro Zuryskie \\*
Ajsze, Baszak, Fatmie, Dżanan, Burczu i Raszidzie. \\*
Ich mężowie, śliczni chłopcy, pięciu braci, jeden obcy: \\*
Simon, Lukas, Christian, Tobias, Jonas i Andreas. \\*
Dostał tata Turek zdjęcia, każde z podobizną zięcia. \\*
Porwał tatę szał \\*
Włosy z głowy rwał \\[\zwrotkaspace]

\refrenspace Cyniczne córy Zurychu\ldots \\[\zwrotkaspace]
 
Pracują wszystkie razem, z Rosją handlują gazem \\*
Ajsze, Baszak, Fatma, Dżanan, Burczu i Raszida. \\*
Statkami ślą przez Bosfor uran, miedź i fosfor \\*
Ajsze, Baszak, Fatma, Dżanan, Burczu i Raszida. \\*
W tajemnicy przed mężami handlują też wyrzutniami \\*
Ajsze, Baszak, Fatma, Dżanan, Burczu i Raszida. \\*
Tata zaś u życia kresu został gwiazdą show-bussinesu, \\*
Radio ,,Stambuł 2'' \\*
Na okrągło gra: \\[\zwrotkaspace]
 
\refrenspace Cyniczne córy Zurychu\ldots \\[\zwrotkaspace]

\end{piosenka_dluga} 