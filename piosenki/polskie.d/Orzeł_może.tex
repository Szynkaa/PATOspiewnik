\begin{piosenka}{Orzeł może -- Artur Andrus}

Możesz nosić okulary, & E \\
Możesz kontraktować zboże, & E \\
Czesać się jak Zygmunt Stary, & A \\
Jesteś orzeł. Orzeł może. & E \\
Spodnie możesz nosić szersze, & cis Gis \\
A na bluzce wilkołaki, & A E \\
Ręcznie przepisywać wiersze, & A E \\
O, na przykład taki: & H \\[\zwrotkaspace]

\refrenspace Człowieku! No przecież, & E A \\
\refrenspace Jak sam nie chcesz, to się nie ciesz, & E E$^7$ \\
\refrenspace Ale niech cię tak nie peszą & A A$^7$ \\
\refrenspace Ci, co się do ciebie cieszą, & E \\
\refrenspace Chudzi, grubi, & H \\
\refrenspace Starzy, młodzi, & A \\
\refrenspace Daj się lubić, & E \\
\refrenspace Co ci szkodzi? & E \\[\zwrotkaspace]

Możesz wracać z biblioteki & E \\
O nieprzyzwoitej porze, & E \\
Regulować brwi i rzeki, & A \\
Jesteś orzeł. Orzeł może. & E \\
Latem rzucać się na siano, & cis Gis \\
Żeby sprawdzić jak się gniecie, & A E \\
Możesz też o czwartej rano & A E \\
Śpiewać w internecie & H \\[\zwrotkaspace]

\refrenspace Człowieku! No przecież\ldots \\[\zwrotkaspace]

Możesz poznać w leśnej głuszy, & E \\
Miłych państwa z Państwa Środka, & E \\
Jeśli z domu się nie ruszysz, & A \\
Trudno będzie kogoś spotkać. & E \\
Możesz poczuć w sercu kłucie, & cis Gis \\
Może zrobić się przyjemnie, & A E \\
Tylko raz się do mnie uciesz, & A E \\
Uśmiechnij się ze mnie. & H \\[\zwrotkaspace]

\refrenspace Człowieku! No przecież\ldots \\[\zwrotkaspace]

\end{piosenka}