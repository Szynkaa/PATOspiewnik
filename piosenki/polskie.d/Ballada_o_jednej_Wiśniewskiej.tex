\begin{piosenka_dluga}{Ballada o jednej Wiśniewskiej -- Jarema Stępowski}
	
Żyli w pałacu hrabia z hrabiną, & a \\*
On zwał się Rodryg, ona Francesca, & d a \\ *
A w drugim domku za ich meliną & d a \\*
Mieszkała sobie jedna Wiśniewska. & a E a \\[\zwrotkaspace]

Niewinne serce miała hrabina & a \\*
I takąż duszę pieską, niebieską,  & d a \\ *
A on był gałgan i straszna świnia, & d a \\ *
Bo pitigrilił się z tą Wiśniewską. & a E a \\[\zwrotkaspace]

Biedna hrabina łzami płakała, & a \\*
Z ciągłej żałoby wyschła na deskę & d a \\* 
I na kolanach męża błagała: & d a \\ *
Odczep się, draniu, od tej Wiśniewskiej. & a E a \\[\zwrotkaspace]

Próżno chodziła z hrabią na udry, & a \\*
Na próżno klęła swą dolę pieską, & d a \\ *
On ciągle ganiał do tej łachudry & d a \\ *
I szeptał czule: ,,O, ty Wiśniewsko!''! & a E a \\[\zwrotkaspace]

Aż raz hrabina miecz zdjęła z ściany, & a \\*
Zmierzyła hrabię okiem królewskim. & d a \\ *
Siedź tu, powiada, ty -- w herb drapany, & d a \\ *
Dzisiaj nie pójdziesz do tej Wiśniewskiej. & a E a \\[\zwrotkaspace]

On zaś będący pod alkoholem, & a \\*
Czyli, jak mówią -- zalany w pestkę, & d a \\ *
Wyrżnął hrabinę łbem w antresolę & d a \\ *
I dawaj, gazu, do tej Wiśniewskiej! & a E a \\[\zwrotkaspace]

Biedna hrabina padła na dywan, & a \\*
Cała zalała się krwią niebieską, & d a \\ *
A gdy poczuła, że dogorywa, & d a \\ *
Rzekła: poczekaj, o ty, Wiśniewsko! & a E a \\[\zwrotkaspace]

Potem się odbył pogrzeb wspaniały, & a \\*
Hrabia nad grobem uronił łezkę, & d a \\ *
Strasznie się martwił przez dzionek cały, & d a \\ *
A na noc poszedł\ldots \ do tej Wiśniewskiej. & a E a \\[\zwrotkaspace]

Wtedy hrabina z mogiły wstała, & a \\*
Wyrwała z trumny sękatą deskę, & d a \\ *
Poszła za hrabią, na śmierć go sprała & d a \\ *
I rozwaliła łeb tej Wiśniewskiej. & a E a \\[\zwrotkaspace]

Chociaż lebiegi grzeszyli tyle & a \\*
I na nich w końcu też przyszła kreska. & d a \\ *
Dziś sobie leżą w jednej mogile: & d a \\ *
Hrabia, hrabina i\ldots \ ta Wiśniewska.	& a E a \\[\zwrotkaspace]
	
\end{piosenka_dluga}