\begin{piosenka}[4mm]{Beskid -- Andrzej Wierzbicki}
A w Beskidzie rozzłocony buk & G C D G C D \\
A w Beskidzie rozzłocony buk & G C a D \\
Będę chodził Bukowiną & C D \\
Z dłutem w ręku & G \\
By w dziewczęcych twarzach uśmiech rzeźbić & C G \\
Niech nie płaczą już & C D \\
Niech się cieszą po kapliczkach moich dróg & C D G C D \\[\zwrotkaspace]

\refrenspace W Beskidzie malowany cerkiewny dach & G C D G \\
\refrenspace W Beskidzie zapach miodu w bukowych pniach & G C H$^7$ e \\
\refrenspace Tutaj wracam, gdy ruda jesień & C D \\
\refrenspace Na przełęcze swój tobół niesie & G C \\
\refrenspace Słucham bicia dzwonów w przedwieczorny czas & G C D \\[\zwrotkaspace]

\refrenspace W Beskidzie malowany wiatrami dom & G C D G \\
\refrenspace W Beskidzie -- tutaj słowa inaczej brzmią & G C H$^7$ e \\
\refrenspace Kiedy krzyczę w jesienną ciszę & C D \\
\refrenspace Kiedy wiatrem szeleszczą liście & G C \\
\refrenspace Kiedy wolność się tuli w ciepło moich rąk & G C D \\
\refrenspace Gdy jak źrebak się tuli do mych rąk & C D G \\[\zwrotkaspace]

A w Beskidzie zamyślony czas & G C D G C D \\
A w Beskidzie zamyślony czas & G C a D \\
Będę chodził z nim & C \\
Poddaszem gór & D \\
By zerwanych marzeń struny przywiązywać & G C G \\
Niespokojnym dłoniom drzew & C D \\
Niech mi grają na rozstajach moich dróg & C D G C D \\[\zwrotkaspace]

\refrenspace W Beskidzie\ldots \\
\end{piosenka}
