\begin{piosenka}{Rzeka -- WGB}
Wsłuchany w twą cichą piosenkę & C F$^{7+}$ C F$^{7+}$ \\
Wyszedłem nad brzeg pierwszy raz & C F$^{7+}$ e e \\
Wiedziałem już, rzeko, że kocham cię, rzeko & F F e a \\
Że odtąd pójdę z tobą & F e d G \\[\zwrotkaspace]

\refrenspace O, dobra rzeko & C F$^{7+}$ C F$^{7+}$ \\
\refrenspace O, mądra wodo & C F$^{7+}$ e e \\
\refrenspace Wiedziałaś, gdzie stopy znużone prowadzić & F F e a \\
\refrenspace Gdy sił już było brak & F e d G \\
\refrenspace Brak & C F$^{7+}$ C F$^{7+}$ \\[\zwrotkaspace]

Wieże miast, łuny świateł & C F$^{7+}$ C F$^{7+}$ \\
Ich oczy zszarzałe nie raz & C F$^{7+}$ e e \\
Witały mnie pustką, żegnały milczeniem & F F e a \\
Gdym stał się twoim nurtem & F e d G \\[\zwrotkaspace]

\refrenspace O, dobra rzeko\ldots \\[\zwrotkaspace]

Po dziś dzień z tobą, rzeko & C F$^{7+}$ C F$^{7+}$ \\
Gdzież począł, gdzie kres dał ci Bóg & C F$^{7+}$ e e \\
Ach, życia mi braknie, by szlak twój przemierzyć & F F e a \\
By poznać twą melodię & F e d G \\[\zwrotkaspace]

\refrenspace O, dobra rzeko\ldots \\
\end{piosenka}
\\[2cm]
\chord{t}{x,p3,p3,p2,p1,n}{F$^{7+}$}
