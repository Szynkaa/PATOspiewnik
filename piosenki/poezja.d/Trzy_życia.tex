\begin{piosenka}{Trzy życia -- Andrzej Korycki \& Dominika Żukowska}
\akordy{E A E}\\
\akordy{E A H E}\\
\akordy{H}\\[\zwrotkaspace]

Pewnie kiedyś mnie spytają & E A E \\
Czy naprawdę warto było & E A E \\
Całe życie obcym ludziom śpiewać & E \\
Obcym, tak jak wy\ldots & E H$^7$ \\
A ja powiem wtedy cicho & E A E \\
Gdyby tak się przytrafiło & E A E \\
Mieć trzy życia, to bym mogła dla was & E \\
Śpiewać życia trzy. & E H$^7$ E \\[\zwrotkaspace]

\refrenspace A to pierwsze prześpiewałabym & A \\
\refrenspace Za chatkę jakąś marną, & E \\
\refrenspace A to drugie prześpiewałabym & A \\
\refrenspace Za czyjeś ciepłe serce, & cis Fis H \\
\refrenspace A to trzecie prześpiewałabym za darmo & E A E \\
\refrenspace No, bo czy mi jeszcze trzeba czegoś więcej. & E H$^7$ E \\[\zwrotkaspace]

A gdy wszystko już przeżyję & E A E \\
I gdy wszystko już wyśpiewam & E A E \\
Zapakuję w worek nuty & E \\
I te dobre, i te złe. & E H$^7$ \\
I gdy w czarnym futerale & E A E \\
Wyląduję w środku nieba & E A E \\
To zaśpiewam trzy piosenki -- & E \\
Może wysłuchają mnie\ldots & E H$^7$ E \\[\zwrotkaspace]

\refrenspace A tę pierwszą to zaśpiewam im & A \\
\refrenspace Za chatkę jakąś marną, & E \\
\refrenspace A tę drugą to zaśpiewam im & A \\
\refrenspace Za czyjeś ciepłe serce, & cis Fis H \\
\refrenspace $\|$ I tej trzeciej nie zaśpiewam już za darmo -- & E A E \\
\refrenspace Niech wam za nią tu spokoju ześlą więcej. $\|\times2$ & E H$^7$ E \\
\end{piosenka}