\begin{piosenka}{Ku chwale astronomii pieśń -- Małgorzata Pluskota}
 
Dziwny list dostałem, dziwna w nim treść & D A \\
,,Astronomiczny obóz'' -- cokolwiek to jest & G h A \\
Fizyka w środku lasu, ogniska i śpiew & D G \\
A nocą patrzeć w gwiazdy -- dlaczego by nie? & h A \\[\zwrotkaspace]

\refrenspace Więc niech lato trwa, & G A \\
\refrenspace Gdy głód wiedzy w nas & D \\
\refrenspace A serca, jak niebo, pełne gwiazd & G A h \\
\refrenspace Niech z całych sił śpiewana & G A \\
\refrenspace Dumnie brzmi wśród drzew & D h \\
\refrenspace Ku chwale astronomii pieśń! & G A D \\[\zwrotkaspace]

Kondycję mam już niezłą po kółkach ze stu & D A \\
Na wszystkich obsach jestem, choć brakuje snu & G h A \\
Zadania z olimpiady na wartach się śnią & D G \\
A Drogi Mlecznej światła nad obozem lśnią & h A \\[\zwrotkaspace]

\refrenspace Więc niech lato trwa\ldots \\[\zwrotkaspace]

Choć było to już dawno i list przepadł gdzieś & D A \\
Co roku w środek lasu coś wciąż ciągnie mnie & G h A \\
A tam, pośród przyjaciół, przybyła znów twarz & D G \\
I ktoś, jak ja przed laty, pomyślał pierwszy raz: & h A \\[\zwrotkaspace]

\refrenspace Więc niech lato trwa\ldots

\end{piosenka}