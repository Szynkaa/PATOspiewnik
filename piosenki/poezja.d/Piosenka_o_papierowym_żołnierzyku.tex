\begin{piosenka}{Piosenka o papierowym żołnierzyku -- Bułat Okudżawa}
	
Raz pewien żołnierz sobie żył & a \\
Odważny i zawzięty. & a G \\
Lecz cóż -- zabawką tylko był, & G \\
Z papieru był wycięty. & G a \\[\zwrotkaspace]

Choć zmieniać świat i zwalczać zło \\
Niezmiennie był gotowy, \\
Stał ciągle wśród zabawek, bo \\
Był przecież papierowy. \\[\zwrotkaspace]

I w ogień gotów był jak w dym \\
Pójść za was bez namowy, \\
I mieliśmy sto pociech z nim -- \\
Był przecież papierowy. \\[\zwrotkaspace]

Lecz nie ujawniał przed nim sztab \\
Tajemnic swych wojskowych. \\
A czemu tak? A temu tak, \\ 
Że był on papierowy. \\[\zwrotkaspace]

Wyzywał los, w pogardzie miał \\
Tchórzliwych maruderów \\
I ,,Ognia! Ognia!'' -- ciągle łkał, \\
A przecież był z papieru. \\[\zwrotkaspace]

Niejeden wódz już w ogniu znikł, \\
Niejeden szeregowy -- \\
I poszedł w ogień, zginął w mig \\
Żołnierzyk papierowy. \\

\end{piosenka}