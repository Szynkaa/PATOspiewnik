\begin{piosenka}{Majster Bieda -- WGB}

\akordy{C F e d G C} \\[\zwrotkaspace]

Skąd przychodził, kto go znał & C F \\
Kto mu rękę podał, kiedy & C F G \\
Nad rowem siadał, wyjmował chleb & C G \\
Serem przekładał i dzielił się z psem & e a \\
Tyle wszystkiego, co z sobą miał & G F e d \\
Majster Bieda & G C F e d G C \\[\zwrotkaspace]

Czapkę z głowy ściągał, gdy & C F \\
Wiatr gałęzie chylił drzewom & C F G \\
Śmiał się do słońca i śpiewał do gwiazd & C G \\
Drogę bez końca, co przed nim szła & e a \\
Znał jak pięć palców, jak szeląg zły & G F e d \\
Majster Bieda & G C F e d G C \\[\zwrotkaspace]

Nikt nie pytał, skąd się wziął & C F \\
Gdy do ognia się przysiadał & C F G \\
Wtulał się w krąg ciepła jak w kożuch & C G \\
Znużony drogą wędrowiec Boży & e a\\
Zasypiał długo, gapiąc się w noc & G F e d \\
Majster Bieda & G C F e d G C \\[\zwrotkaspace]

Aż nastąpił taki rok & C F \\
Smutny rok, tak widać trzeba & C F G \\
Nie przyszedł Bieda zieloną wiosną & C G \\
Miejsce, gdzie siadał, zielskiem zarosło & e a \\
I choć niejeden wytężał wzrok & G F \\
Choć lato pustym gościńcem przeszło & G F \\
Z rudymi liśćmi jesieni schedą & G F \\
Wiatrem niesiony popłynął w przeszłość & G F \\
Wiatrem niesiony popłynął w przeszłość & G F \\
Wiatrem niesiony popłynął w przeszłość & G F G \\
Majster Bieda & C F e d G C \\[\zwrotkaspace]

\end{piosenka}\\