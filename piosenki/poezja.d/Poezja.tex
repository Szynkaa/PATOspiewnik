\begin{piosenka_dluga}{Poezja -- Władysław Broniewski, Na Bani}

Ty przychodzisz jak noc majowa, & cis gis \\*
Biała noc, uśpiona w jaśminie & A H \\*
I jaśminem pachną twoje słowa, & cis gis \\*
I księżycem sen srebrny płynie, & A H \\[\zwrotkaspace]

Płyniesz cicha przez noce bezsenne &cis gis \\*
-- cichą nocą tak liście szeleszczą -- & A H \\*
Szepcesz sny, szepcesz słowa tajemne & cis gis \\*
W słowach cichych skąpana jak w deszczu\ldots & A H \\[\zwrotkaspace]

To za mało! Za mało! Za mało! & cis gis \\*
Twoje słowa tumanią i kłamią! & A H \\*
Piersiom żywych daj oddech zapału, & cis gis \\*
Wiew szeroki i skrzydła do ramion! & A H \\[\zwrotkaspace]

Nam te słowa ciche nie starczą. & cis gis \\*
Marne słowa. I błahe. I zimne. & A H \\*
Ty masz werbel nam zagrać do marszu! & cis gis \\*
Smagać słowem! Bić pieśnią! Wznieść hymnem! & A H \\[\zwrotkaspace]

Jest gdzieś radość ludzka, zwyczajna, & cis gis \\*
Jest gdzieś jasne i piękne życie. & A H \\*
Powszedniego chleba słów daj nam & cis gis \\*
I stań przy nas, i rozkaż - bić się! & A H \\[\zwrotkaspace]

Niepotrzebne nam białe westalki, & cis gis \\*
Noc nie zdławi świętego ognia -- & A H \\*
Bądź jak sztandar rozwiany wśród walki, & cis gis \\*
Bądź jak w wichrze wzniesiona pochodnia! & A H \\[\zwrotkaspace]

Odmień, odmień nam słowa na wargach, & cis gis \\*
Naucz śpiewać płomienniej i prościej, & A H \\*
Niech nas miłość ogromna potarga. & cis gis \\*
Więcej bólu i więcej radości! & A H \\[\zwrotkaspace]

Jeśli w pięści potrzebna ci harfa, & cis gis \\*
Jeśli harfa ma zakląć pioruny, & A H \\*
Rozkaż żyły na struny wyszarpać & cis gis \\*
I naciągać, i trącać jak struny. & A H \\[\zwrotkaspace]

Trzeba pieśnią bić aż do śmierci, & cis gis \\*
Trzeba głuszyć w ciemnościach syk węży. & A H \\*
Jest gdzieś życie piękniejsze od nędzy. & cis gis \\*
I jest miłość. I ona zwycięży. & A H \\[\zwrotkaspace]

Wtenczas daj nam, poezjo, najprostsze & cis gis \\*
Ze słów prostych i z cichych -- najcichsze, & A H \\*
A umarłych w wieczności rozpostrzyj & cis gis \\*
Jak chorągwie podarte na wichrze. & A H \\[\zwrotkaspace]

\end{piosenka_dluga}
