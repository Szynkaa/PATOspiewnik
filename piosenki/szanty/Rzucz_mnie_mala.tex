\begin{piosenka}[6mm]{Rzuć mnie mała -- Andrzej Korycki}

Rzuć mnie mała, a będę miał pretekst, & D \\
By zaraz w morze iść. & E$^7$ \\ 
Rzuć mnie mała, by naszła znów mnie chęć & A A$^7$ \\
Na morskich przygód kiść & D E$^7$ A \\[\zwrotkaspace]
	
Rzuć mnie mała, niech inny w twej szafie & D \\
Swój rozpakuje wór & E$^7$ \\ 
Rzuć mnie mała, bo tak jak na rafie tu & A A$^7$ \\
Tkwię, więc tnij ten sznur. & D E$^7$ D \\[\zwrotkaspace]
 
\refrenspace Ja do ciebie, co tu kryć, nie pasuję, (nie pasuję!) & G D \\
\refrenspace Bo ty w pionie lubisz być, a ja się kiwać chcę. & E$^7$ A$^7$ \\
\refrenspace Rzuć mnie mała, niech inny w twej szafie & G \\
\refrenspace Swój rozpakuje wór & E$^7$ \\ 
\refrenspace Rzuć mnie mała, bo tak jak na rafie tu & A A$^7$ \\
\refrenspace Tkwię, więc tnij ten sznur & D A$^7$ D \\
\refrenspace $\| \times 2$ \\[\zwrotkaspace]

Rzuć mnie mała, bo ja nie potrafię żyć & D \\
Jak lądowy szczur & E$^7$ \\ 
Rzuć mnie mała i z innym wieś zasiedl & A A$^7$ \\
Gdzieś u podnóży gór & D E$^7$ A \\[\zwrotkaspace]

Rzuć mnie mała, a cieszył się będę & D \\
Od rufy aż po dziób & E$^7$ \\ 
Rzuć mnie mała, i zrób to czym prędzej, & A A$^7$ \\
Bo jeśli nie - to ślub! & D A$^7$ D \\[\zwrotkaspace]

\refrenspace Choć do ciebie, co tu kryć, nie pasuję, (nie pasuję!) & G D \\
\refrenspace Bo ty w pionie lubisz być, a ja się kiwać chcę. & E$^7$ A$^7$ \\
\refrenspace Zgódź się mała, a cieszył się będę & G \\
\refrenspace Od rufy aż po dziób & E$^7$ \\ 
\refrenspace Zgódź się mała, i zrób to czym prędzej, & A A$^7$ \\
\refrenspace A wtedy jutro ślub! & D A$^7$ D \\
\refrenspace $\| \times 2$ \\[\zwrotkaspace]

\end{piosenka}