\begin{piosenka}{Gdzieś tam -- Jerzy Porębski}

Gdzieś tam, na krańcach Wielkiej Wody, & D cis Fis \\
Gdzieś tam, gdzie giną ludzkie drogi, & h C D$^7$ \\
Gdzieś tam, gdzie siedzi stary, siwy Bóg, & G fis \\
Jest dom naszych wszystkich dusz. & e A$^7$ D \\[\zwrotkaspace]

Gdzieś tam, w syntezie wszelkich Światów, & D cis Fis \\
Gdzieś tam, bez wiary, bez dogmatów, & h C D$^7$ \\
Wśród naturalnych praw, & G fis \\
Do załatwienia mam parę spraw. & e A$^7$ D \\[\zwrotkaspace]

\refrenspace Po pierwsze, po drugie, po trzecie, po czwarte\ldots & Fis h E$^7$ \\
\refrenspace I co to życie było warte? & E$^7$ A$^7$ \\[\zwrotkaspace]

Gdzieś tam, gdzie nieśmiertelne żyje, & D cis Fis \\
Gdzieś tam, gdzie źródło prawdy bije, & h C D$^7$ \\
Gdzie zło i dobro traci sens, & G fis \\
W ostatni popłynę rejs. & e A$^7$ D \\[\zwrotkaspace]

Gdzieś tam, na drugą stronę cienia, & D cis Fis \\
Gdzieś tam, gdzie duch materię zmienia, & h C D$^7$ \\
Przejmę ostatnią z wacht, & G fis \\
Do kei przytulę jacht. & e A$^7$ D \\[\zwrotkaspace]

Gdzieś tam, na krańcach Wielkiej Wody, & D cis Fis \\
Gdzieś tam, gdzie giną ludzkie drogi, & h C D$^7$ \\
Gdzieś tam, gdzie siedzi stary, siwy Bóg, & G fis \\
Jest dom naszych wszystkich dusz\ldots & e A$^7$ D \\[\zwrotkaspace]

\end{piosenka}