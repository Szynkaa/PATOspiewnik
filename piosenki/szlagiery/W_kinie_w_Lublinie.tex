{\small \begin{piosenka}{W kinie w Lublinie -- Brathanki}

\akordy{D e Fis h}\\
\akordy{Cis Cis Fis h}\\[\zwrotkaspace]

O świcie i o zmroku $\|\times2$ & h \\
W południe, w nocy o świcie & Fis h \\
W Skarżysku i w Sanoku $\|\times2$ & h \\
Ty mnie pokochaj nad życie & Fis h \\[\zwrotkaspace]

W berecie, w czapce, chustce $\|\times2$ & cis \\
W czapce od stryjka ze Lwowa & Gis cis \\
Na falochronie w Ustce $\|\times2$ & cis \\
Zakochaj we mnie się znowu & Gis cis \\[\zwrotkaspace]

\refrenspace W kinie, w Lublinie -- kochaj mnie & D A \\
\refrenspace W maju, w tramwaju -- kochaj mnie & h Fis \\
\refrenspace Nie marudź, nie szlochaj & G D \\
\refrenspace Ale z całej siły kochaj & e Fis h \\
\refrenspace W gminie, w Kętrzynie -- kochaj mnie & D A \\
\refrenspace W metrze i w swetrze -- kochaj mnie & h Fis \\
\refrenspace Czy miasto czy wiocha, & G D \\
\refrenspace ty mnie z całej siły kochaj & e Fis h \\[\zwrotkaspace]

W radości no i w smutku $\|\times2$ & h \\
W radości z ciepłego lata & Fis h \\
Na piasku plaży, w Gródku $\|\times2$ & h \\
Kochaj mnie do końca świata & Fis h \\[\zwrotkaspace]

W spokoju oraz w gniewie $\|\times2$ & cis \\
W spokoju palmowych niedziel & Gis cis \\
W Marwałdzie i w Gętlewie $\|\times2$ & cis \\
Kochaj w bogactwie i w biedzie & Gis cis \\[\zwrotkaspace]

\refrenspace W kinie, w Lublinie \ldots \\[\zwrotkaspace]

Jak młody ułan dzielnie $\|\times2$ & h \\
Jak wartki na wiosnę strumień & Fis h \\
Na nartach wodnych w Mielnie $\|\times2$ & h \\
Kochaj najmocniej jak umiesz & Fis h \\[\zwrotkaspace]

Latem w przydrożnym rowie $\|\times2$ & cis \\
Zimą na sankach i nartach & Gis cis \\
Najmocniej zaś w Krakowie $\|\times2$ & cis \\
Kochaj bom tego jest warta & Gis cis \\[\zwrotkaspace]

\refrenspace W kinie, w Lublinie \ldots \\[\zwrotkaspace]

\end{piosenka}}
