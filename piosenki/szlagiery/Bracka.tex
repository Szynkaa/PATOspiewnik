\begin{piosenka_dluga}{Bracka -- Grzegorz Turnau}

Na północy ściął mróz & gis Fis \\*
Z nieba spadł wielki wóz & H E \\*
Przykrył drogi pola i lasy & G D a \\*
Myśli zmarzły na lód & D G \\*
Dobre sny zmorzył głód & H e \\*
Lecz przynajmniej się można przestraszyć & F E \\[\zwrotkaspace]

Na południu już skwar & gis Fis \\*
Miękki puch z nieba zdarł & H E \\*
Kruchy pejzaż na piasek przepalił & G D a \\*
Jak upalnie mój boże & D G \\*
Lecz przynajmniej być może & H e \\*
Wreszcie byśmy się tam zakochali & F E \\[\zwrotkaspace]
 
\akordy{a G d e} \\[\zwrotkaspace]

\refrenspace A w krakowie & a G \\*
\refrenspace Na brackiej pada deszcz & F G \\*
\refrenspace Gdy konieczność istnienia & F G \\*
\refrenspace Trudna jest do zniesienia & d B \\*
\refrenspace W korytarzu & a G \\*
\refrenspace I w kuchni pada też & F G \\*
\refrenspace Przyklejony do ściany & F G \\*
\refrenspace Zwijam mokre dywany & d B \\*
\refrenspace Nie od deszczu mokre & a G \\*
\refrenspace Lecz od łez & F G \\[\zwrotkaspace]

Na zachodzie już noc & gis Fis \\*
Wciągasz głowę pod koc & H E \\*
Raz zasypiasz i sprawa jest czysta & G D a \\*
Dłonie zapleć i złóż & D G \\*
Nie obudzisz się już & H e \\*
Lecz przynajmniej raz możesz się wyspać & F E \\[\zwrotkaspace]

Jeśli wrażeń cię głód & gis Fis \\*
Zagna kiedyś na wschód & H E \\*
Nie za długo tam chyba wytrzymasz & G D a \\*
Lecz na wschodzie przynajmniej & D G \\*
Życie płynie zwyczajnie & H e \\*
Słońce wschodzi i dzień się zaczyna & F E \\[\zwrotkaspace]

\refrenspace A w Krakowie & a G \\*
\refrenspace Na brackiej pada deszcz & F G \\*
\refrenspace Przemęczony i senny & F G \\*
\refrenspace Zlew przecieka kuchenny & d B \\*
\refrenspace Kaloryfer& a G \\*
\refrenspace Jak mysz się poci też & F G \\*
\refrenspace Z góry na dół kałuże & F G \\*
\refrenspace Przepływają po sznurze & d B \\*
\refrenspace Nie od deszczu mokrym & a G \\*
\refrenspace Lecz od łez & F G \\[\zwrotkaspace]

\refrenspace Bo w Krakowie & a G \\*
\refrenspace Na brackiej pada deszcz & F G \\*
\refrenspace Gdy zagadka istnienia & F G \\*
\refrenspace Zmusza mnie do myślenia & d B \\*
\refrenspace W korytarzu & a G \\*
\refrenspace I w kuchni pada też & F G \\*
\refrenspace Przyklejony do ściany & F G \\*
\refrenspace Zwijam mokre dywany & d B \\*
\refrenspace Nie od deszczu mokre & a G \\*
\refrenspace Lecz od łez & F G \\[\zwrotkaspace]
 
Bo w krakowie & a G \\*
Na brackiej pada deszcz & F G \\[\zwrotkaspace]

\end{piosenka_dluga}