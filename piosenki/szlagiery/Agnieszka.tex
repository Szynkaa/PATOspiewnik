{\small \begin{piosenka}[1mm]{Agnieszka -- Łzy}

Było ciepłe lato, choć czasem padało & F C \\*
Dużo wina się piło i mało się spało & d B \\*
Tak zaczęła się wakacyjna przygoda & C d \\*
On był jeszcze młody i ona była młoda & B \\[\zwrotkaspace]

Zakochani przy świetle księżyca nocami & F C \\*
Chodzili długimi, leśnymi ścieżkami & d B \\*
Tak mijały tygodnie, lecz rozstania nadszedł czas & C d \\*
Zawsze mówił jedno zdanie: ,,Moje śliczne ty kochanie'' & B \\[\zwrotkaspace]

Ostatniego dnia tych pamiętnych wakacji & F C \\*
Kochali się namiętnie w męskiej ubikacji & d B \\*
I przysięgli przed Bogiem miłość wzajemną & C d \\*
Że za rok się spotkają i na zawsze ze sobą już będą & B F B C \\[\zwrotkaspace]

Tęsknił za nią i pisał do niej listy miłosne & F C \\*
W samotności przeżył jesień, zimę, wiosnę & d B \\*
Nie wytrzymał do wakacji, postanowił ją odwiedzić & C d \\*
Bo nie dostał już dawno od niej żadnej odpowiedzi & B \\[\zwrotkaspace]

Gdy przyjechał do jej domu po dość długiej podróży & F C \\*
Cieszył się, że ją zobaczy, w końcu tyle dla niej znaczył & d B \\*
Lecz gdy ona go ujrzała, szybko się schowała & C d \\*
Drzwi mu matka otworzyła i tak mu powiedziała & B \\[\zwrotkaspace]

\refrenspace Agnieszka już dawno tutaj nie mieszka & F B C F d C \\*
\refrenspace $\Vert\ \times$ 4 \\[\zwrotkaspace]

Rozczarował się, bo takie są zawody miłosne & F C \\*
Cierpiał całą jesień, zimę, no i wiosnę & d B \\*
A gdy przeszło mu zupełnie, pojechał na wakacje & C d \\*
W tamto miejsce, by zobaczyć tę pamiętną ubikację & B \\[\zwrotkaspace]

Tak się stało, że przypadkiem ona też tam była & F C \\*
Ucieszyła się ogromnie, gdy go tylko zobaczyła & d B \\*
Zapytała się, czy w sercu jego jest jeszcze Agnieszka & C d \\*
Odpowiedział jednym zdaniem: ,,Moje śliczne ty kochanie'' & B \\[\zwrotkaspace]

\refrenspace Agnieszka już dawno\ldots \\*
\end{piosenka} }
