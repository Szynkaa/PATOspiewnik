\begin{piosenka}{Córka rybaka -- Wały Jagiellońskie}

Gdy księżyc świecił na niebie dla ciebie & C G C \\
Poczułem miłość co przyszła jak wiatr & C G \\
Me serce było w gorącej potrzebie & G \\
Córką rybaka ty byłaś ja -- góral z Tatr & G C \\
Jelenie gdzieś nad jeziorem sennie ryczały & C G C \\
Ryby w jeziorze już poszły dawno spać & C$^7$ F \\
Rzekłaś wtedy do mnie Mój Mały! & F C /C/H/B/A/\\
Cóż ci mogę w te parną, mazurską noc dać & F G C \\ [\zwrotkaspace]
 
\refrenspace Córko rybaka, Mazura z Mazur & C G \\ 
\refrenspace Popatrz jaki na jeziorze wonny glazur & G C C$^7$ \\
\refrenspace Daj mi swe usta, weź mnie w ramiona & F C /C/H/B/A/ \\
\refrenspace Niech się przekonam ile słodyczy & F G \\
\refrenspace Jest w słowie Ilona & C \\ [\zwrotkaspace]

Lato minęło lecz uczucie ogniem płonie & C G C \\
Choć odległość dziś tak wielka dzieli nas & C G \\
Ciągle czuję na mym ciele twoje dwie dłonie & G \\
W uszach moich szumi woda, szemrze las & G C \\
Zakopane całe śniegiem zasypane & C G C \\
A ty piszesz: na jeziorze gruba kra & C$^7$ F \\
Przesyłasz całuski i dwie rybie łuski & F C /C/H/B/A/ \\
Zima minie, lato złączy serca dwa & F G C \\ [\zwrotkaspace]

\refrenspace Córko rybaka\ldots \\

\end{piosenka}