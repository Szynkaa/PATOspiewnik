{\small \begin{piosenka}{Szanta narciarska -- Artur Andrus}

Nazywali go marynarz, & d C d \\
Bo opaskę miał na oku. & d C F \\
Na każdym stoku dziewczyna, & g d \\
Dziewczyna na każdym stoku. & d A d \\
Pochodzi spod Poznania, & d C d \\
Podobno umie wróżyć z kart. & d C F \\
Panny rwie na wiązania, & g d \\
Mężatki -- na długość nart. & B A d \\[1mm]

\refrenspace Caryco mokrego śniegu, & A d \\
\refrenspace Ratrakiem płynę do ciebie pod prąd\ldots \ Hej! & A B \\
\refrenspace Dobrze, że stoisz na brzegu, & g d \\
\refrenspace Bo ja właśnie schodzę na ląd. & B A d \\[1mm]

Nigdy się nie lękał biedy & d C d \\
I się nie przejmował jutrem. & d C F \\
A jego ratrak był kiedyś & g d \\
Zwyczajnym rybackim kutrem. & d A d \\
I woził dorsze i śledzie, & d C d \\
Zimą i latem, okrągły rok. & d C F \\
Teraz, jak nieraz przejedzie & g d \\
Rybami czuć cały stok. & B A d \\[1mm]

\refrenspace Caryco mokrego śniegu\ldots \\[1mm]

Wszyscy w porcie odetchnęli. & d C d \\
Zwiał, nim się zakończył sezon. & d C F \\
Jeszcze się tam jak żagiel bieli & g d \\
Jego czarny kombinezon. & d A d \\
Odpłynął pod Ustrzyki & d C d \\
I przez kobiety wpadł w kłopoty. & d C F \\
Forsę z polowań na orczyki & g d \\
Przehulał na antybiotyk. & B A d \\[1mm]

\refrenspace Caryco mokrego śniegu\ldots \\[1mm]

Jeśli kiedyś go zobaczysz & d C d \\
Na ratraku w podłym świecie, & d C F \\
To powiedz mu, że w Karpaczu & g d \\
Czekają na niego dzieci. & d A d \\
I kiedy opuszcza statek, & d C d \\
Żeby się znowu oddać złu, & d C F \\
Każda z dwudziestu siedmiu matek & g d \\
Dzieciątku śpiewa do snu: & B A d \\[1mm]

\end{piosenka} }